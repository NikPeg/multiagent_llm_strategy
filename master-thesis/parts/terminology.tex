\begin{terminologyList}
    \term[Языковая модель (Language Model, LM)]{алгоритмическая система, обученная предсказывать и генерировать текст на естественном языке}

    \term[Большая языковая модель (Large Language Model, LLM)]{языковая модель значительного размера, способная генерировать когерентный текст и выполнять различные языковые задачи}

    \term[Оркестрация языковых моделей]{процесс координации нескольких языковых моделей или компонентов для последовательного выполнения сложных задач}

    \term[Мультиагентная система]{система, состоящая из нескольких взаимодействующих интеллектуальных агентов, каждый из которых выполняет определенную функцию}

    \term[RAG (Retrieval-Augmented Generation)]{метод улучшения генерации текста языковыми моделями путем предварительного извлечения релевантной информации из внешних источников}

    \term[Галлюцинации LLM]{явление, при котором языковая модель генерирует фактически неверную информацию, представляя её как достоверную}

    \term[Военно-политические игры (ВПИ, Military-Political Games, WPG)]{жанр текстовых стратегических игр, в которых игроки управляют государствами, политическими фракциями или организациями, взаимодействуя посредством письменных приказов и получая ответные вердикты от мастера игры}

    \term[Судья (вердер)]{человек, наделенный полномочиями определять реакцию внешнего мира на действия игрока и описывать в текстовом формате итоги приказов, аналог гейм-мастера}

    \term[Приказ]{сформулированная в текстовой форме воля игрока, использующая находящиеся под контролем игрока силы для преобразования внешнего мира в рамках установленных игрой правил}

    \term[Верд (вердикт)]{текстовое описание результатов выполнения приказов игроков, составляемое вердером}

    \term[Сеттинг]{совокупность особенностей среды, в рамках которой протекает игра, включая историю мира, технологический уровень, географию и культурные особенности}

    \term[Сессия]{игровой процесс, проходящий в рамках одной «игровой реальности», непрерывный процесс отыгрыша в определённой вселенной без удаления игроков и обнуления прогресса}

    \term[Рестарт]{перезапуск проекта и смена игровой сессии}

    \term[Калькулятор]{принцип организации игровой механики, при котором взаимодействие с миром осуществляется через численные переменные и формулы, обеспечивающие определенный уровень автоматизации}

    \term[Ролеплей (РП)]{принцип организации игровой механики, при котором взаимодействие с игровой реальностью осуществляется через прямое текстовое взаимодействие игрока и судьи}

    \term[Классическая ВПИ]{проект, где в рамках игровой механики предусмотрен отыгрыш правителя государства с возможностью действий во всех сферах государственной политики}

    \term[Командно-штабная игра (КШИ)]{разновидность ВПИ с акцентом на военной составляющей, где игроки делятся на команды, отыгрывающие офицерский состав заранее определенных армий}

    \term[Виртуальное государство]{поджанр ВПИ, концентрирующийся на отыгрыше политики в рамках одного государства с детализацией механик принятия государственных решений}

\end{terminologyList}
