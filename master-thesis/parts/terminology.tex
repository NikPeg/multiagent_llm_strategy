\begin{terminologyList}
    \term[Абстрактное синтаксическое дерево (АСД, Abstract Syntax Tree, AST)]{одна из форм промежуточного представления программ в виде древовидной структуры}
    \term[Анализ потока данных (Data Flow Analysis, DFA)]{один из основных методов анализа программ, позволяющий определить в каждой точке программы некоторую информацию о данных, которыми оперирует код}
    \term[Байткод]{одна из форм промежуточного представления программ в виде инструкций, которые близки к машинным и могут быть интерпретированы при помощи виртуальной машины}
    \term[Виртуальная машина Java (Java Virtual Machine, JVM)]{основная часть исполняющей системы Java, исполняющая байткод, полученный из исходного кода программы, на конкретной платформе путём трансляции байткода в машинные инструкции}
    \term[Граф потока управления (ГПУ, Control Flow Graph, CFG)]{множество всех возможных путей выполнения программы, представленное в виде графа}
    \term[Промежуточное представление (Intermediate Representation, IR)]{структура данных или код, используемый внутри компилятора или виртуальной машины для представления программ}
    \term[Статический анализ кода]{анализ исходного кода на предмет ошибок и недочётов без непосредственного выполнения анализируемых программ}
    \term[Common Vulnerabilities and Exposures (CVE)]{база данных общеизвестных уязвимостей информационной безопасности}
    \term[Common Weakness Enumeration (CWE)]{общий перечень и система классификации слабых мест и уязвимостей программного обеспечения}
    \term[Java]{строго типизированный объектно-ориентированный язык программирования общего назначения, разработанный компанией Sun Microsystems}
    \term[Kotlin]{статически типизированный, объектно-ориентированный язык программирования, работающий поверх Java Virtual Machine и разрабатываемый компанией JetBrains}
\end{terminologyList}
