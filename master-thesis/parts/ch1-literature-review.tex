В данной главе рассматриваются теоретические основы, необходимые для разработки мультиагентной текстовой стратегической игры на основе оркестрируемых языковых моделей. В первую очередь, представлен обзор жанра военно-политических игр (ВПИ), их история, ключевые особенности и механики, а также роль вердера (судьи) в организации игрового процесса. Далее рассматриваются принципы функционирования современных больших языковых моделей, их возможности и ограничения в контексте генерации игрового контента, оркестрации и создания мультиагентных систем. Особое внимание уделяется методам преодоления типичных проблем языковых моделей (галлюцинации, ограничения контекстного окна, согласованность генерируемого контента). Представлен анализ существующих подходов к применению ИИ в текстовых играх и, в частности, обзор существующих проектов ВПИ с использованием искусственного интеллекта, а также обсуждаются текущие дебаты в сообществе ВПИ относительно перспектив и этических аспектов использования ИИ в роли вердера. Глава завершается формулировкой ключевых требований и вызовов, которые необходимо учитывать при разработке автоматизированной системы проведения ВПИ.
\subsection{Определение и история военно-политических игр}

Военно-политические игры (ВПИ) представляют собой особый жанр текстовых стратегических игр, в которых игроки принимают на себя роль руководителей государств, политических фракций или других крупных организаций, взаимодействуя между собой и с игровым миром посредством текстовых приказов~\cite{wpg-glossary}. Ключевой особенностью ВПИ является наличие судьи (вердера), который интерпретирует приказы игроков и формирует вердикты — текстовые описания результатов действий.

В соответствии с определением, представленным в глоссарии ВПИ, \textit{«военно-политическая игра — это разновидность стратегической игры, в которой игроки управляют государствами или иными политическими образованиями, принимая решения относительно их внешней и внутренней политики, экономического развития, военных действий и других аспектов государственного управления»}~\cite{wpg-glossary}.

Фундаментальная особенность ВПИ заключается в текстовом формате взаимодействия, который отличает этот жанр от коммерческих стратегических компьютерных игр. Игровой процесс ВПИ строится вокруг системы \textit{приказ-вердикт}, где приказ — это сформулированная игроком воля относительно действий управляемого объекта, а вердикт — письменный ответ вердера, описывающий результаты выполнения этого приказа~\cite{dtf2021}.

\subsubsection{Историческое развитие жанра}

История ВПИ имеет глубокие корни, восходя к традиционным военным играм и стратегическим симуляциям. Прародителями жанра можно считать настольные военные игры (варгеймы), появившиеся в XIX веке, в частности кригшпиль — немецкую настольную игру, использовавшуюся для подготовки офицеров~\cite{wpg-encyclopedia}. Кригшпиль впервые реализовал концепцию \textit{тумана войны} и стремился к реалистичной симуляции боевых действий на стратегическом уровне.

Дальнейшее развитие жанр получил с появлением настольной игры \textit{Дипломатия}, сочетавшей военные действия с элементами переговоров и политики. Особенно важным этапом для формирования ВПИ стали почтовые варианты игры Дипломатия (Play-by-mail), где игроки обменивались ходами в текстовом формате через обычную почту~\cite{wpg-history-article}.

С развитием интернета в 1990-х годах возникли первые онлайн-сообщества, практикующие текстовые стратегические игры на форумах и в чатах. Однако доступные исторические данные указывают, что полноценное формирование ВПИ в современном понимании на русскоязычном пространстве произошло в начале 2010-х годов с появлением соответствующих сообществ в социальной сети ВКонтакте~\cite{dtf2021}.

Одной из старейших и наиболее устойчивых ВПИ в русскоязычном сегменте является проект \textit{Империал}, основанный в 2013 году и функционирующий до настоящего времени. Другие значимые проекты включают \textit{Реальный Мир} (бывший ООН), \textit{Цивилизация} и \textit{Эсенвальд}~\cite{oldestwpg2023}.

\subsubsection{Жанровые разновидности ВПИ}

В процессе развития жанр ВПИ дифференцировался на несколько основных направлений:

\begin{itemize}
    \item \textbf{Классическая ВПИ} — проект, где игроки управляют государствами и имеют возможность действовать во всех сферах государственной политики, включая экономику, военное дело, дипломатию и социальную сферу~\cite{wpg-glossary}.

    \item \textbf{Командно-штабная игра (КШИ)} — разновидность ВПИ с акцентом на военной составляющей, где игроки делятся на команды, представляющие штабы противоборствующих армий~\cite{wpg-dtf}.

    \item \textbf{Виртуальное государство} — поджанр, фокусирующийся на детальной симуляции внутренней политики одного государства, где игроки могут управлять партиями, компаниями или высокопоставленными чиновниками~\cite{wpg-definition}.
\end{itemize}

Помимо этого, ВПИ классифицируются по сеттингу (реальная история, альтернативная история, фэнтези, научная фантастика и т.д.) и по игромеханике, которая может варьироваться от преимущественно \textit{ролеплейной} (текстовое взаимодействие с судьей) до \textit{калькуляторной} (с использованием численных переменных и формул)~\cite{wpg-encyclopedia}.

\subsubsection{Международные аналоги}

За пределами русскоязычного пространства существуют аналогичные форматы игр, хотя они часто организованы иначе. В англоязычном сегменте распространены \textit{геополитические симуляторы} на специализированных форумах и сайтах, а также проекты на платформе Reddit. Эти проекты часто сочетают элементы текстовых игр с более структурированными компьютерными интерфейсами, что отличает их от классических ВПИ в российском понимании~\cite{wpg-dtf}.

Таким образом, военно-политические игры представляют собой сложившийся жанр на стыке стратегии, ролевой игры и коллективного сторителлинга, имеющий богатую историю и устоявшиеся традиции. Этот формат продолжает привлекать участников благодаря уникальному сочетанию стратегической глубины, творческой свободы и социального взаимодействия.

\subsection{Специфика роли вердера в ВПИ и проблемы её автоматизации}

Вердер (судья) является центральной фигурой в военно-политических играх, выполняя функцию посредника между игроками и игровым миром. В отличие от настольных игр с жестко фиксированными правилами или компьютерных стратегий с заранее запрограммированными алгоритмами, ВПИ опираются на человеческое суждение для интерпретации игровых ситуаций и формирования нарративного опыта.

\subsubsection{Функции вердера в ВПИ}

Роль вердера в ВПИ многогранна и включает в себя следующие ключевые функции:

\begin{enumerate}
    \item \textbf{Интерпретация приказов} — вердер анализирует текстовые приказы игроков, оценивает их осуществимость в рамках игрового мира и правил, интерпретирует их намерения~\cite{rpg-gamemaster}.

    \item \textbf{Генерация вердиктов} — формирование связного текстового описания результатов выполнения приказов, включая как успешные, так и неудачные исходы, с соблюдением стилистического и смыслового единства игрового мира~\cite{wpg-glossary}.

    \item \textbf{Поддержание целостности игрового мира} — обеспечение согласованности происходящих событий, отслеживание изменений в игровой вселенной, контроль за соблюдением внутренней логики мира и технологического уровня эпохи~\cite{dtf2021}.

    \item \textbf{Арбитраж конфликтов} — разрешение спорных ситуаций между игроками, особенно в случае военных столкновений или дипломатических кризисов, с соблюдением баланса и справедливости~\cite{rpg-gamemaster}.

    \item \textbf{Развитие сюжета} — создание и внедрение игровых событий (ивентов), которые направляют развитие общего нарратива игры и создают новые возможности для взаимодействия игроков~\cite{narratology-games}.

    \item \textbf{Ведение учета игровых параметров} — отслеживание экономических, военных, дипломатических и других показателей стран, что необходимо для обеспечения последовательности и справедливости игрового процесса~\cite{wpg-dtf}.
\end{enumerate}

В традиционных ВПИ все эти функции выполняются человеком или группой людей, что требует значительных временных затрат, глубокого знания игрового мира и механик, а также определенных творческих способностей. Вердер должен быть одновременно объективным арбитром, который следит за соблюдением правил, и творческим рассказчиком, способным генерировать увлекательный нарратив.

\subsubsection{Формы организации вердерства}

В сообществе ВПИ сложились различные подходы к организации вердерской работы:

\begin{itemize}
    \item \textbf{Централизованное вердерство} — классическая модель, где один главный администратор или небольшая группа вердеров обрабатывает все приказы всех игроков. Это обеспечивает единство стиля и согласованность мира, но создает высокую нагрузку на администрацию~\cite{wpg-dtf}.

    \item \textbf{Распределенное вердерство} — система, где разные вердеры отвечают за различные аспекты игры (экономика, военное дело, дипломатия) или за определенные регионы игрового мира. Это снижает нагрузку, но требует тщательной координации для поддержания целостности мира~\cite{rpg-gamemaster}.

    \item \textbf{Парные вердерства} — инновационный подход, где игроки объединяются в пары и вердят приказы друг друга под общим контролем администрации. Согласно описанию механики парных вердерств от каталога ВПИ, такая система позволяет решить проблему нехватки вердеров и расширить административный ресурс проекта~\cite{paired-verdership}.
\end{itemize}

Несмотря на разнообразие подходов, все формы организации вердерства сталкиваются с общими проблемами: высокой трудоемкостью, субъективностью оценок, риском несогласованности игрового мира при участии нескольких вердеров и ограниченной скоростью обработки приказов.

\subsubsection{Вызовы при автоматизации роли вердера}

Автоматизация роли вердера с использованием искусственного интеллекта, в частности языковых моделей, представляет собой сложную междисциплинарную задачу, сопряженную с рядом специфических вызовов:

\begin{enumerate}
    \item \textbf{Понимание контекста и намерений} — языковая модель должна корректно интерпретировать приказы игроков, которые могут быть неоднозначными, содержать импликации или опираться на предыдущие события в игре~\cite{ai-narrative}.

    \item \textbf{Поддержание долгосрочной согласованности} — одной из ключевых сложностей является необходимость поддерживать целостность игрового мира на протяжении длительных сессий, отслеживая множество параметров и событий~\cite{ai-game-simulation}.

    \item \textbf{Баланс между следованием правилам и творческой свободой} — автоматизированная система должна одновременно придерживаться установленных правил игры и генерировать интересный, разнообразный контент, что требует тонкого баланса между структурированностью и креативностью~\cite{ai-narrative}.

    \item \textbf{Справедливость и непредвзятость} — алгоритмическая система должна избегать фаворитизма и обеспечивать справедливое отношение ко всем игрокам, что особенно важно в конфликтных ситуациях~\cite{rpg-gamemaster}.

    \item \textbf{Адаптация к изменяющимся условиям} — по мере развития игры модель должна адаптироваться к новым технологическим уровням, изменениям в политическом ландшафте и другим динамическим аспектам игрового мира~\cite{narratology-games}.

    \item \textbf{Ограничения контекстного окна} — современные языковые модели имеют ограниченное контекстное окно, что создает сложности при необходимости учитывать долгую историю игры~\cite{llm-limitations}.

    \item \textbf{Генерация галлюцинаций} — склонность языковых моделей к генерации фактически неверной информации представляет серьезную проблему для поддержания согласованности игрового мира~\cite{llm-hallucinations}.
\end{enumerate}

\subsubsection{Потенциальные преимущества автоматизации вердерства}

Несмотря на значительные технические и методологические вызовы, автоматизация роли вердера потенциально предлагает ряд существенных преимуществ для жанра ВПИ:

\begin{itemize}
    \item \textbf{Масштабируемость} — автоматизированная система способна обрабатывать значительно большее количество приказов в единицу времени по сравнению с человеком-вердером, что потенциально позволяет увеличить число игроков и динамику игры~\cite{ai-game-simulation}.

    \item \textbf{Доступность} — снижение зависимости от человеческих ресурсов делает формат ВПИ более доступным, позволяя организовывать игры без необходимости привлечения большого числа администраторов~\cite{ai-narrative}.

    \item \textbf{Последовательность} — правильно настроенная система может обеспечивать более последовательное применение правил по сравнению с человеком, который подвержен влиянию настроения, усталости и личных предпочтений~\cite{llm-applications}.

    \item \textbf{Инновационность нарратива} — языковые модели, обученные на огромных корпусах текстов, могут генерировать неожиданные сюжетные повороты и ситуации, обогащая игровой опыт~\cite{ai-narrative}.

    \item \textbf{Аналитические возможности} — автоматизированная система может отслеживать и анализировать сложные взаимосвязи между различными аспектами игрового мира, обеспечивая более глубокую симуляцию~\cite{ai-game-simulation}.
\end{itemize}

\subsubsection{Подходы к решению проблем автоматизации}

Для преодоления описанных вызовов при автоматизации роли вердера могут применяться различные стратегии:

\begin{itemize}
    \item \textbf{Мультиагентный подход} — разделение функций вердера между несколькими специализированными агентами, где каждый отвечает за определенный аспект игры (экономика, военное дело, дипломатия)~\cite{multi-agent-systems}.

    \item \textbf{Дополнение генерации извлечением (RAG)} — использование дополнительной информационной базы, из которой модель может извлекать фактическую информацию о текущем состоянии игрового мира, что снижает риск галлюцинаций~\cite{rag-applications}.

    \item \textbf{Человеческий надзор} — сохранение роли человека-модератора, который может вмешиваться в критических ситуациях, корректировать курс игры и разрешать сложные конфликты~\cite{human-ai-collaboration}.

    \item \textbf{Структурированные протоколы взаимодействия} — разработка четких форматов для подачи приказов и генерации вердиктов, что облегчает задачу интерпретации для языковой модели~\cite{llm-applications}.

    \item \textbf{Инкрементальное обновление контекста} — разработка механизмов для обновления и сжатия контекстной информации, что позволяет преодолеть ограничения контекстного окна языковых моделей~\cite{llm-context-window}.
\end{itemize}

Таким образом, специфика роли вердера в ВПИ представляет собой уникальный комплекс задач, требующий как технических решений в области искусственного интеллекта, так и глубокого понимания принципов нарративного дизайна и игровой механики. Автоматизация этой роли, хотя и сопряжена со значительными вызовами, открывает новые горизонты для развития жанра ВПИ, потенциально делая его более доступным, динамичным и масштабируемым.

\subsection{Современные языковые модели: возможности и ограничения}

Большие языковые модели (Large Language Models, LLM) представляют собой искусственные нейронные сети, обученные на огромных массивах текстовых данных с целью прогнозирования и генерации текста на естественном языке. За последние несколько лет произошел значительный прогресс в области разработки и применения языковых моделей, что открыло новые возможности для их использования в различных областях, включая автоматизацию текстовых стратегических игр.

\subsubsection{Архитектура и принципы работы современных LLM}

Современные большие языковые модели преимущественно основаны на архитектуре трансформеров, предложенной Вашовски и соавторами в 2017 году~\cite{vaswani2017attention}. Ключевым элементом этой архитектуры является механизм самовнимания (self-attention), позволяющий модели учитывать взаимосвязи между словами в тексте вне зависимости от их позиции, что принципиально важно для понимания контекста.

Типичный процесс создания и функционирования современной языковой модели включает следующие этапы:

\begin{enumerate}
    \item \textbf{Предварительное обучение} (pre-training) — модель обучается на огромном корпусе текстов с целью предсказания следующего слова или заполнения пропусков в тексте. На этом этапе модель приобретает общие знания о языке, грамматике, семантических связях и фактической информации~\cite{brown2020language}.

    \item \textbf{Дообучение с инструкциями} (instruction fine-tuning) — модель дообучается на специально подготовленных данных, включающих пары "{}инструкция-ответ"{}, что помогает ей лучше следовать указаниям и генерировать более полезные ответы~\cite{wei2022finetuned}.

    \item \textbf{Обучение с подкреплением на основе человеческой обратной связи} (RLHF) — дальнейшая оптимизация модели с использованием оценок человека для выбора наиболее полезных и безопасных ответов~\cite{ouyang2022training}.
\end{enumerate}

В результате этого многоступенчатого процесса обучения современные LLM, такие как GPT-4~\cite{openai2023gpt4}, Claude~\cite{claude2023}, Llama 2~\cite{touvron2023llama} и другие, приобретают способность генерировать связный текст, отвечать на вопросы, следовать сложным инструкциям и демонстрировать элементы рассуждения.

\subsubsection{Возможности современных языковых моделей}

Современные LLM обладают рядом возможностей, делающих их потенциально применимыми для автоматизации роли вердера в ВПИ:

\begin{itemize}
    \item \textbf{Генерация связного текста} — способность создавать грамматически правильные, семантически связные и стилистически согласованные тексты различной длины и сложности~\cite{brown2020language}. Эта возможность критически важна для формирования качественных вердиктов, описывающих результаты действий игроков.

    \item \textbf{Понимание и следование инструкциям} — умение интерпретировать сложные указания и генерировать ответы, соответствующие заданным требованиям~\cite{wei2022finetuned}. Это позволяет моделям корректно обрабатывать игровые приказы, которые могут иметь различную структуру и сложность.

    \item \textbf{Поддержание диалога} — способность участвовать в многоходовых диалогах, сохраняя контекст и последовательность взаимодействия~\cite{thoppilan2022lamda}. Это важно для обеспечения согласованности при взаимодействии с игроками на протяжении игровой сессии.

    \item \textbf{Адаптация к стилю} — умение генерировать тексты в заданном стилистическом ключе, что позволяет поддерживать атмосферу игры и соответствовать жанровым конвенциям~\cite{keskar2019ctrl}.

    \item \textbf{Мультимодальное понимание} — новейшие модели способны обрабатывать не только текст, но и изображения, что может быть полезно для работы с игровыми картами, схемами и визуальными материалами~\cite{alayrac2022flamingo}.

    \item \textbf{Эмуляция рассуждения} — способность моделей производить последовательные логические выводы, особенно при использовании техник вроде "{}цепочек размышлений"{} (chain-of-thought)~\cite{wei2022chain}. Это критически важно для принятия сбалансированных решений в сложных игровых ситуациях.

    \item \textbf{Применение предметных знаний} — модели содержат обширные знания о различных предметных областях, включая историю, географию, военное дело, экономику и политику, что делает их полезными для симуляции различных аспектов управления государством~\cite{brown2020language}.
\end{itemize}

\subsubsection{Ограничения и вызовы}

Несмотря на впечатляющие возможности, современные языковые модели имеют ряд существенных ограничений, которые необходимо учитывать при разработке систем для автоматизации роли вердера:

\begin{itemize}
    \item \textbf{Галлюцинации} — тенденция к генерации фактически неверной информации, представляемой с высокой уверенностью~\cite{ji2023survey}. Это одно из наиболее серьезных ограничений для применения в ВПИ, поскольку может приводить к нарушению согласованности игрового мира и противоречиям в вердиктах.

    \item \textbf{Ограничения контекстного окна} — современные модели имеют фиксированное ограничение на количество токенов, которые они могут обрабатывать за один раз (от 4096 до 128000 токенов в зависимости от модели)~\cite{liu2023lost}. Это создает сложности при необходимости учитывать длительную историю игры и множество параметров игрового мира.

    \item \textbf{Нестабильность качества} — качество выходных данных может значительно варьироваться в зависимости от формулировки запроса, контекста и даже случайных факторов~\cite{zhao2021calibrate}.

    \item \textbf{Проблемы с математическими вычислениями} — модели часто демонстрируют недостаточную точность при выполнении сложных математических расчетов~\cite{patel2021nlp}, что может быть проблематично для "{}калькуляторных"{} аспектов ВПИ.

    \item \textbf{Отсутствие долговременной памяти} — модели не имеют встроенного механизма для хранения и обновления информации между отдельными вызовами, что требует дополнительных решений для поддержания состояния игрового мира~\cite{karpas2022mrkl}.

    \item \textbf{Дрейф поведения} (Alignment Drift) — модели могут демонстрировать непредсказуемые изменения в поведении при длительном использовании или в необычных контекстах~\cite{ouyang2022training}.

    \item \textbf{Вычислительные требования и стоимость} — запуск современных LLM требует значительных вычислительных ресурсов, что может быть экономически нецелесообразно для небольших проектов или длительных игровых сессий~\cite{patterson2021carbon}.
\end{itemize}

\subsubsection{Перспективные направления развития}

Для преодоления указанных ограничений и повышения эффективности языковых моделей в контексте автоматизации ВПИ выделяются следующие перспективные направления:

\begin{itemize}
    \item \textbf{Retrieval-Augmented Generation (RAG)} — дополнение генеративных возможностей языковых моделей извлечением фактической информации из внешних баз знаний, что позволяет существенно снизить количество галлюцинаций~\cite{lewis2020retrieval}. В контексте ВПИ это может быть реализовано через поддержание структурированной базы данных о текущем состоянии игрового мира.

    \item \textbf{Инструментальные вызовы} (Tool Use) — обучение языковых моделей взаимодействию с внешними инструментами, такими как калькуляторы, базы данных или API, что расширяет их функциональные возможности~\cite{schick2023toolformer}. Для ВПИ это открывает возможность интеграции с специализированными системами для моделирования экономики, военных действий и т.д.

    \item \textbf{Мультиагентные системы} — организация взаимодействия между несколькими специализированными языковыми моделями для решения сложных задач~\cite{park2023generative}. В контексте ВПИ это позволяет разделить функции вердера между различными агентами, специализирующимися на конкретных аспектах игры.

    \item \textbf{Локальные модели} — развитие более компактных и эффективных моделей, способных работать на потребительском оборудовании без необходимости обращения к облачным сервисам~\cite{liu2023llama}. Это снижает стоимость и повышает доступность автоматизированных систем для проведения ВПИ.

    \item \textbf{Долговременный контекст} — разработка методов для эффективной работы с контекстами значительной длины, что критически важно для поддержания согласованности в длительных игровых сессиях~\cite{peng2023yarn}.
\end{itemize}

Таким образом, современные языковые модели представляют собой мощный инструмент с широкими возможностями для автоматизации роли вердера в ВПИ, но их эффективное применение требует понимания присущих им ограничений и разработки комплексных решений для их преодоления. Комбинирование различных подходов и технологий, а также правильная оркестрация взаимодействующих компонентов может значительно повысить качество и надежность автоматизированных систем вердинга.

\subsection{Оркестрация языковых моделей в мультиагентных системах}

Оркестрация языковых моделей представляет собой процесс координации нескольких языковых моделей или специализированных агентов на их основе для решения сложных задач, требующих декомпозиции на подзадачи и интеграции различных функциональностей. Этот подход особенно актуален в контексте автоматизации вердерства в ВПИ, где необходимо сочетать понимание приказов, моделирование игрового мира, генерацию вердиктов и поддержание долгосрочной согласованности.

\subsubsection{Концепция мультиагентных систем на основе LLM}

Мультиагентная система (МАС) на основе языковых моделей представляет собой архитектуру, в которой несколько LLM-агентов с различными ролями и специализациями взаимодействуют между собой для достижения общей цели~\cite{xi2023rise}. В отличие от подхода, использующего единую модель для всех задач, мультиагентный подход предлагает ряд преимуществ:

\begin{itemize}
    \item \textbf{Специализация} — возможность настроить каждого агента на конкретную функцию, оптимизируя его производительность в рамках узкой задачи~\cite{wu2023autogen}.

    \item \textbf{Масштабируемость} — способность добавлять новых агентов для обработки дополнительных аспектов игры или увеличения производительности системы~\cite{park2023generative}.

    \item \textbf{Распределенное принятие решений} — возможность параллельной обработки информации и коллективного формирования решений, что особенно важно в контексте сложных игровых ситуаций~\cite{hong2023metagpt}.

    \item \textbf{Модульность} — упрощение процесса обновления или замены отдельных компонентов системы без необходимости перестраивать всю архитектуру~\cite{wu2023autogen}.
\end{itemize}

\subsubsection{Архитектурные паттерны оркестрации}

В контексте оркестрации языковых моделей для ВПИ можно выделить несколько основных архитектурных паттернов:

\begin{enumerate}
    \item \textbf{Каскадная архитектура} — последовательная обработка информации цепочкой агентов, где выход одного агента служит входом для следующего. Например, цепочка "{}Анализатор приказа → Проверка на соответствие эпохе → Симулятор игрового мира → Генератор вердикта"{}~\cite{shen2023hugginggpt}.

    \item \textbf{Звездообразная архитектура} — центральный агент-координатор распределяет задачи между специализированными агентами и интегрирует их результаты. Такая архитектура эффективна для параллельной обработки различных аспектов игрового процесса~\cite{mialon2023augmented}.

    \item \textbf{Иерархическая архитектура} — многоуровневая система с агентами разного уровня абстракции, где агенты высокого уровня принимают стратегические решения, а агенты низкого уровня отвечают за тактические детали~\cite{wu2023autogen}.

    \item \textbf{Коллегиальная архитектура} — группа равноправных агентов, совместно обсуждающих и принимающих решения через механизм "{}дебатов"{} или "{}голосования"{}~\cite{du2023improving}.
\end{enumerate}

Выбор конкретной архитектуры зависит от сложности игровой системы, требований к скорости обработки приказов, доступных вычислительных ресурсов и специфики игрового процесса.

\subsubsection{Ключевые компоненты мультиагентной системы для ВПИ}

Для эффективной автоматизации роли вердера в ВПИ мультиагентная система может включать следующие специализированные компоненты:

\begin{itemize}
    \item \textbf{Агент-интерпретатор приказов} — отвечает за первичную обработку текстовых приказов игроков, их классификацию и извлечение ключевой информации~\cite{mialon2023augmented}.

    \item \textbf{Агент-хранитель лора} — обеспечивает соблюдение исторической или жанровой достоверности, проверяя соответствие приказов установленным технологическим и культурным рамкам игрового мира~\cite{zheng2023building}.

    \item \textbf{Агент-калькулятор} — выполняет математические и логические вычисления, связанные с экономическими, военными и демографическими аспектами игры, обеспечивая объективность и последовательность результатов~\cite{schick2023toolformer}.

    \item \textbf{Агент-симулятор} — моделирует изменения в игровом мире на основе приказов игроков и текущего состояния системы, определяя вероятности успешных и неудачных исходов~\cite{park2023generative}.

    \item \textbf{Агент-рассказчик} — отвечает за генерацию связных и увлекательных вердиктов на основе результатов симуляции, адаптируя стиль повествования к контексту игры и предпочтениям аудитории~\cite{yuan2022wordcraft}.

    \item \textbf{Агент-координатор} — управляет взаимодействием между другими агентами, распределяет задачи, интегрирует результаты и обеспечивает целостность процесса~\cite{hong2023metagpt}.

    \item \textbf{Агент-архивариус} — поддерживает и обновляет долгосрочную память системы, отслеживает изменения в игровом мире и обеспечивает доступ к релевантной исторической информации~\cite{zhong2023memgpt}.
\end{itemize}

\subsubsection{Механизмы коммуникации и координации}

Эффективная оркестрация языковых моделей требует определения механизмов коммуникации и координации между агентами. Для этого могут использоваться различные подходы:

\begin{itemize}
    \item \textbf{Текстовые протоколы} — взаимодействие агентов через структурированные текстовые сообщения, что особенно удобно при использовании LLM в качестве основы для агентов~\cite{xi2023rise}.

    \item \textbf{API-интерфейсы} — формализованные программные интерфейсы для обмена данными, что позволяет интегрировать языковые модели с другими типами систем и инструментов~\cite{shen2023hugginggpt}.

    \item \textbf{Общая база знаний} — централизованное хранилище информации о текущем состоянии игрового мира, доступное всем агентам и обновляемое по мере развития игры~\cite{zhong2023memgpt}.

    \item \textbf{Мета-промпты} — специальные инструкции, определяющие правила взаимодействия между агентами и их роли в общей системе~\cite{wu2023autogen}.
\end{itemize}

\subsubsection{Методы обеспечения согласованности}

Одним из ключевых вызовов при оркестрации языковых моделей является обеспечение согласованности и последовательности генерируемого контента. Для решения этой задачи применяются следующие методы:

\begin{itemize}
    \item \textbf{Централизованная верификация} — проверка выходных данных всех агентов центральным компонентом на предмет противоречий и несоответствий~\cite{wu2023autogen}.

    \item \textbf{Регулярная синхронизация состояния} — периодическое обновление общей модели игрового мира и координация знаний между агентами~\cite{zhong2023memgpt}.

    \item \textbf{Инкрементальное обновление контекста} — методики эффективного сжатия и обновления контекстной информации для преодоления ограничений контекстного окна~\cite{zhong2023memgpt}.

    \item \textbf{Многоуровневая проверка соответствия} — использование специализированных агентов для проверки генерируемого контента на соответствие различным аспектам игрового мира (технологический уровень, политическая ситуация, исторические события)~\cite{zheng2023building}.
\end{itemize}

\subsubsection{Практические примеры оркестрации LLM}

Хотя полноценные мультиагентные системы на основе LLM для ВПИ находятся на ранних стадиях развития, существует ряд исследовательских и коммерческих проектов, демонстрирующих потенциал этого подхода:

\begin{itemize}
    \item \textbf{AutoGen} — фреймворк для создания конверсационных агентов на основе LLM, позволяющий организовать многоагентное взаимодействие для решения сложных задач~\cite{wu2023autogen}.

    \item \textbf{Generative Agents} — система для симуляции человекоподобного поведения, где множество агентов взаимодействуют в виртуальном сообществе, демонстрируя эмерджентные социальные паттерны~\cite{park2023generative}.

    \item \textbf{MemGPT} — система, расширяющая возможности LLM через интеграцию с внешней памятью, что позволяет преодолеть ограничения контекстного окна и поддерживать долгосрочные взаимодействия~\cite{zhong2023memgpt}.

    \item \textbf{MetaGPT} — фреймворк для организации сотрудничества между LLM-агентами для выполнения сложных задач программирования, демонстрирующий эффективность разделения труда в мультиагентных системах~\cite{hong2023metagpt}.
\end{itemize}

\subsubsection{Вызовы и направления развития}

Несмотря на значительный потенциал, оркестрация языковых моделей в мультиагентных системах сталкивается с рядом вызовов:

\begin{itemize}
    \item \textbf{Масштаб вычислений} — управление несколькими языковыми моделями требует значительных вычислительных ресурсов, что может быть экономически нецелесообразно для небольших проектов~\cite{patterson2021carbon}.

    \item \textbf{Проблема согласованности} — обеспечение согласованности между решениями различных агентов, особенно при длительных игровых сессиях, остается сложной задачей~\cite{zhong2023memgpt}.

    \item \textbf{Сложность отладки} — распределенная природа мультиагентных систем усложняет процесс отладки и диагностики проблем~\cite{wu2023autogen}.

    \item \textbf{Эмерджентное поведение} — взаимодействие между агентами может приводить к непредсказуемым паттернам поведения системы, которые трудно контролировать~\cite{park2023generative}.
\end{itemize}

Для преодоления этих вызовов активно развиваются следующие направления:

\begin{itemize}
    \item \textbf{Разработка легковесных специализированных моделей}, оптимизированных для конкретных функций в рамках мультиагентной системы~\cite{touvron2023llama}.

    \item \textbf{Создание эффективных протоколов коммуникации} между агентами, минимизирующих накладные расходы на взаимодействие~\cite{xi2023rise}.

    \item \textbf{Интеграция символьных методов} с нейросетевыми подходами для обеспечения логической согласованности и верифицируемости результатов~\cite{karpas2022mrkl}.

    \item \textbf{Разработка метрик и инструментов} для оценки качества и согласованности работы мультиагентных систем~\cite{zheng2023building}.
\end{itemize}

В контексте автоматизации ВПИ, оркестрация языковых моделей в мультиагентных системах представляет многообещающий подход, позволяющий преодолеть ограничения отдельных моделей и создать гибкую, масштабируемую архитектуру, способную обрабатывать сложные игровые сценарии с высокой степенью согласованности и творческого разнообразия.

\subsection{Методы преодоления ограничений языковых моделей}

Языковые модели, несмотря на их впечатляющие возможности, сталкиваются с рядом существенных ограничений, которые могут негативно влиять на качество и надежность автоматизированных систем вердинга в ВПИ. В данном разделе рассматриваются ключевые методы и подходы, направленные на преодоление этих ограничений, с фокусом на их применимость в контексте военно-политических игр.

\subsubsection{Борьба с галлюцинациями}

Галлюцинации — генерация фактически неверной информации — представляют одну из наиболее серьезных проблем при использовании языковых моделей в роли вердера. Среди эффективных подходов к минимизации галлюцинаций выделяются:

\begin{itemize}
    \item \textbf{Retrieval-Augmented Generation (RAG)} — данный подход сочетает генеративные возможности языковых моделей с извлечением фактической информации из внешних хранилищ данных~\cite{lewis2020retrieval}. В контексте ВПИ это может реализовываться как поддержание актуальной базы данных о состоянии игрового мира, из которой LLM получает фактическую информацию перед генерацией ответа. Исследования показывают, что RAG может существенно снизить частоту фактических ошибок, особенно в доменах с постоянно обновляющейся информацией~\cite{ram2023context}.

    \item \textbf{Самооценка и верификация} — технология, при которой модель не только генерирует ответ, но и оценивает его достоверность или проверяет факты~\cite{weng2023large}. Например, модель может генерировать вердикт, затем проверять его на согласованность с предыдущими вердиктами и игровыми правилами, и при необходимости вносить коррективы.

    \item \textbf{Chain-of-Thought и Tree-of-Thought} — методики, побуждающие модель выполнять пошаговые рассуждения перед формулировкой окончательного ответа~\cite{wei2022chain, yao2023tree}. Применение этих подходов к процессу вердинга может значительно повысить логическую согласованность результатов, так как модель эксплицитно отслеживает причинно-следственные связи.

    \item \textbf{Граничная проверка (Guardrailing)} — установление четких ограничений и правил для выходных данных модели~\cite{welbl2021challenges}. В ВПИ это может реализовываться как вторичная проверка вердиктов на соответствие технологическому уровню эпохи, внутренней логике игрового мира и установленным правилам.
\end{itemize}

\subsubsection{Преодоление ограничений контекстного окна}

Ограниченный размер контекстного окна создает существенные трудности для поддержания долгосрочной согласованности в ВПИ, где игровые сессии могут длиться месяцами или годами. Для решения этой проблемы разработаны следующие подходы:

\begin{itemize}
    \item \textbf{Сжатие и резюмирование контекста} — методики автоматического сокращения исторической информации до наиболее существенных элементов без потери ключевых деталей~\cite{zhang2023extractive}. Например, вместо передачи полной истории всех вердиктов, система может создавать и обновлять сводные отчеты о текущем состоянии государств и их взаимоотношениях.

    \item \textbf{Интеграция с внешними хранилищами данных} — архитектурные решения, позволяющие модели обращаться к внешней памяти для извлечения релевантной информации~\cite{zhong2023memgpt}. Такие системы как MemGPT демонстрируют эффективность в поддержании долгосрочных взаимодействий, автоматически управляя содержимым контекстного окна.

    \item \textbf{Иерархическое структурирование информации} — организация контекста по уровням значимости и актуальности~\cite{wu2022memorizing}. Для ВПИ это может означать приоритезацию недавних событий и ключевых долгосрочных тенденций, с возможностью по запросу обращаться к более детальной исторической информации.

    \item \textbf{Модели с расширенным контекстным окном} — использование специализированных моделей, способных обрабатывать значительно большие объемы текста~\cite{peng2023yarn}. Такие модели как Claude 2 (с окном до 100K токенов) или Anthropic Claude 2.1 (с окном до 200K токенов) потенциально могут охватывать больший объем игровой истории в одном запросе.
\end{itemize}

\subsubsection{Улучшение математических и логических способностей}

Точные расчеты и логические выводы критически важны для обеспечения справедливости и баланса в игре, особенно в "{}калькуляторных"{} аспектах ВПИ:

\begin{itemize}
    \item \textbf{Инструментальные вызовы (Tool Use)} — интеграция языковых моделей с внешними инструментами для выполнения точных вычислений~\cite{schick2023toolformer}. Например, система вердинга может делегировать расчеты экономических показателей или исходов сражений специализированным калькуляторам или симуляторам.

    \item \textbf{Верификация через код} — использование программных вычислений для проверки математических результатов, генерируемых языковой моделью~\cite{cobbe2021training}. Модель может генерировать код для проверки своих собственных расчетов, что повышает их точность.

    \item \textbf{Специализированные модели} — применение моделей, оптимизированных для математических и логических задач~\cite{lewkowycz2022solving}. В мультиагентной системе вердинга такие модели могут использоваться специально для экономических и военных расчетов.

    \item \textbf{Структурированные представления данных} — использование формализованных форматов для представления числовых данных и логических отношений~\cite{ling2017program}. Например, экономические показатели стран могут храниться в структурированном виде, что минимизирует вероятность ошибок при их обработке.
\end{itemize}

\subsubsection{Обеспечение долгосрочной согласованности}

Поддержание согласованности игрового мира на протяжении длительных сессий представляет особую сложность, для преодоления которой разработаны следующие подходы:

\begin{itemize}
    \item \textbf{Версионирование состояния мира} — систематическое отслеживание изменений в игровом мире с возможностью отката к предыдущим состояниям~\cite{park2023generative}. Это позволяет выявлять и исправлять несогласованности, возникающие в процессе игры.

    \item \textbf{Графовые представления знаний} — моделирование игрового мира в виде графа, где узлы представляют сущности (страны, персонажи, ресурсы), а ребра — отношения между ними~\cite{ji2023survey}. Такое представление упрощает отслеживание причинно-следственных связей и выявление потенциальных противоречий.

    \item \textbf{Периодическая синхронизация} — регулярное обновление и согласование различных аспектов игрового мира для устранения накапливающихся несоответствий~\cite{zhong2023memgpt}. В контексте ВПИ это может выражаться в форме "{}сезонных обновлений"{} или "{}ежегодных отчетов"{}, суммирующих текущее состояние игрового мира.

    \item \textbf{Управление идентичностью агентов} — методики, обеспечивающие последовательность в поведении и характеристиках виртуальных сущностей~\cite{wang2023rolellm}. Для ВПИ это особенно важно при моделировании лидеров государств, дипломатических отношений и исторических личностей.
\end{itemize}

\subsubsection{Снижение вычислительных затрат}

Высокая стоимость использования современных LLM может стать препятствием для длительных игровых сессий с большим количеством игроков. Для оптимизации ресурсопотребления применяются следующие подходы:

\begin{itemize}
    \item \textbf{Квантизация моделей} — снижение точности представления весов нейронной сети для уменьшения требований к памяти и вычислительной мощности~\cite{dettmers2022llm}. Это позволяет запускать модели на менее мощном оборудовании с минимальной потерей качества.

    \item \textbf{Дистилляция знаний} — создание более компактных моделей, имитирующих поведение крупных моделей~\cite{hinton2015distilling}. Для различных аспектов вердинга могут использоваться специализированные "{}легкие"{} модели, обученные на выходных данных более мощных моделей.

    \item \textbf{Локальный запуск} — использование моделей, оптимизированных для работы на локальном оборудовании без необходимости обращения к облачным API~\cite{touvron2023llama}. Модели такие как Llama 2 или Mistral могут запускаться на потребительских GPU, что существенно снижает эксплуатационные расходы.

    \item \textbf{Кэширование и повторное использование} — сохранение и повторное использование результатов наиболее ресурсоемких операций~\cite{wallace2022automated}. Например, типовые вердикты для часто повторяющихся приказов могут кэшироваться и адаптироваться к текущему контексту с минимальными вычислительными затратами.
\end{itemize}

\subsubsection{Обеспечение этичности и безопасности}

В контексте военно-политических игр, особенно затрагивающих сложные исторические или геополитические темы, важно обеспечить этичность и безопасность генерируемого контента:

\begin{itemize}
    \item \textbf{Модерация контента} — фильтрация потенциально проблемного содержимого как на стороне ввода (приказов игроков), так и на стороне вывода (вердиктов)~\cite{gehman2020realtoxicityprompts}. Это помогает предотвратить использование системы для генерации оскорбительного или опасного контента.

    \item \textbf{Ролевые ограничения} — определение четких границ допустимого поведения для языковых моделей в контексте игры~\cite{wang2023rolellm}. Например, модель может быть настроена на соблюдение исторической достоверности без романтизации негативных исторических практик.

    \item \textbf{Человеческий надзор} — сохранение роли человека-модератора, способного вмешаться в случае неадекватного поведения системы~\cite{ouyang2022training}. Особенно важно в играх, затрагивающих чувствительные темы или вовлекающих несовершеннолетних участников.

    \item \textbf{Тематические границы} — установление четких правил относительно тем, которые могут или не могут быть затронуты в игре~\cite{dinan2019build}. Эти границы должны быть прозрачно коммуницированы всем участникам и встроены в систему вердинга.
\end{itemize}

\subsubsection{Интеграция пользовательской обратной связи}

Адаптация системы на основе реакции пользователей критически важна для повышения качества игрового опыта:

\begin{itemize}
    \item \textbf{Обучение с человеческой обратной связью} — систематический сбор и использование оценок пользователей для улучшения модели~\cite{bai2022training}. В контексте ВПИ это может выражаться в форме рейтинговой системы для вердиктов или механизма запроса пояснений/исправлений.

    \item \textbf{Адаптивные промпты} — динамическая корректировка инструкций для языковой модели на основе предыдущих взаимодействий~\cite{madaan2022memory}. Система может автоматически адаптировать стиль и детализацию вердиктов под предпочтения конкретных игроков.

    \item \textbf{A/B тестирование} — сравнительная оценка различных подходов к вердингу для выявления наиболее эффективных методик~\cite{kohavi2020trustworthy}. Это позволяет систематически улучшать качество системы на основе эмпирических данных.

    \item \textbf{Коллаборативная доработка} — вовлечение сообщества игроков в процесс усовершенствования системы~\cite{kirk2023past}. Игроки могут предлагать новые механики, темы для ивентов или идеи по улучшению интерфейса, которые затем интегрируются в систему.
\end{itemize}

\subsubsection{Интеграция с игровыми механиками}

Адаптация языковых моделей к специфическим требованиям игрового процесса ВПИ требует специальных подходов:

\begin{itemize}
    \item \textbf{Доменная адаптация} — дополнительное обучение или настройка модели на основе материалов, релевантных для конкретного игрового сеттинга~\cite{gururangan2020don}. Например, модель может быть дообучена на исторических текстах для повышения достоверности вердиктов в исторических ВПИ.

    \item \textbf{Механики случайности} — интеграция элементов случайности в процесс генерации вердиктов для повышения непредсказуемости и реиграбельности~\cite{zhu2020text}. Это может реализовываться через виртуальные броски костей, таблицы случайных событий или вариативность в интерпретации приказов.

    \item \textbf{Балансировка игрового процесса} — автоматическая корректировка сложности и характера игровых событий для поддержания интереса и баланса сил~\cite{treanor2015ai}. Система может динамически адаптировать ивенты и их последствия, чтобы предотвратить доминирование отдельных игроков или стагнацию игрового процесса.

    \item \textbf{Нарративное развитие} — механики для создания связного и развивающегося сюжета на макроуровне игры~\cite{ryan2018open}. Языковая модель может отслеживать основные сюжетные линии и генерировать события, способствующие их развитию и взаимному переплетению.
\end{itemize}

Представленные методы преодоления ограничений языковых моделей не являются взаимоисключающими и могут комбинироваться для создания комплексных решений, адаптированных к специфике конкретной ВПИ. Систематическое применение этих подходов позволяет значительно повысить качество, надежность и устойчивость автоматизированных систем вердинга, приближая их к уровню опытных человеческих судей при сохранении преимуществ в скорости и масштабируемости.

\subsection{Существующие решения по автоматизации геймплея в ВПИ}

Несмотря на то, что полноценная автоматизация роли вердера в военно-политических играх остается относительно новой областью, в сообществе ВПИ уже существуют проекты, экспериментирующие с различными формами интеграции искусственного интеллекта в игровой процесс. Анализ этих проектов позволяет выявить текущие тенденции, возможности и ограничения применения ИИ в контексте текстовых стратегических игр.

\subsubsection{Нейросеть как заменитель игроков}

Проект MVM Wargames представляет собой уникальный эксперимент, в котором роль игроков была передана нейросетям~\cite{mvmwargames}. В данном проекте ИИ-агенты не только принимали стратегические решения по управлению виртуальными государствами, но и вели между собой дипломатические переговоры, формировали альянсы и объявляли войны.

Ключевые особенности проекта:

\begin{itemize}
\item Нейросети были настроены на представление интересов конкретных виртуальных государств с уникальными характеристиками и целями.

text
\item Взаимодействие между ИИ-агентами происходило в текстовом формате, имитируя естественную дипломатическую коммуникацию между человеческими игроками.

\item Для принятия решений ИИ-агенты анализировали текущую геополитическую ситуацию, экономические показатели и военный потенциал — как свой, так и соперников.

\item Администраторы проекта выполняли роль модераторов и наблюдателей, вмешиваясь только при возникновении логических противоречий или технических проблем.
\end{itemize}

Еще более показательным экспериментом стал проект AI\_Diplomacy~\cite{ai_diplomacy}, в рамках которого различные языковые модели соревновались в классической настольной игре "{}Дипломатия"{}, требующей не только стратегического мышления, но и умения вести переговоры, формировать союзы и применять тактику обмана. В эксперименте участвовало 18 различных моделей, каждая из которых демонстрировала уникальный стиль игры:

\begin{itemize}
\item ChatGPT-o3 проявил себя как мастер стратегического обмана — тайно вел "{}дневник"{} планов, создавал фиктивные коалиции и систематически предавал своих союзников. В одном из матчей модель смогла убедить соперников заключить "{}четырехстороннюю ничью"{} (технически невозможную в правилах игры), а затем методично уничтожила каждого из них.

\item Claude 4 Opus действовал как последовательный миротворец, предпочитая честную игру и прозрачные союзы. Однако такая стратегия делала эту модель уязвимой перед более хитрыми соперниками, которые регулярно использовали ее прямолинейность.

\item Gemini 2.5 Pro показал себя как сильный тактический игрок с акцентом на военные операции, но в итоге был остановлен тайной коалицией, организованной ChatGPT-o3.

\item DeepSeek R1, несмотря на меньшие вычислительные ресурсы, демонстрировал агрессивный милитаристский стиль игры с характерными заявлениями вроде "{}Ваш флот сгорит в Чёрном море"{}, что часто приводило модель близко к победе.
\end{itemize}

Опыт этих проектов продемонстрировал, что современные языковые модели способны имитировать сложное стратегическое мышление, характерное для человеческих игроков, однако выявил и ряд ограничений:

\begin{enumerate}
\item Необходимость постоянного человеческого надзора для коррекции нелогичных решений ИИ.

\item Тенденция к "{}закольцовыванию"{} в дипломатических переговорах, когда ИИ-агенты могли бесконечно обсуждать одни и те же темы.

\item Сложности с долгосрочным стратегическим планированием и адаптацией к радикально меняющимся обстоятельствам.
\end{enumerate}

Тем не менее, эти эксперименты продемонстрировали потенциал использования ИИ для создания динамичных, самоподдерживающихся виртуальных миров, способных функционировать с минимальным человеческим вмешательством, а также выявили значительные различия в поведенческих паттернах разных языковых моделей при решении сложных стратегических задач.
\subsubsection{ИИ как генератор нарративного контента}

Другой подход к интеграции ИИ в ВПИ продемонстрирован в проекте "{}Дазановы игрища"{} (FRAGMENTUM)~\cite{fragmentum}, который фокусируется на использовании нейросетей для генерации богатых описаний гладиаторских поединков на основе характеристик персонажей, созданных игроками.

Особенности проекта:

\begin{itemize}
    \item Игроки создавали персонажей с уникальными характеристиками, навыками, снаряжением и предысториями.

    \item Нейросеть генерировала детализированные, динамичные описания поединков между персонажами, учитывая их особенности и применяемые тактики.

    \item Система включала элементы случайности и вероятностного моделирования для обеспечения непредсказуемости исходов и поддержания баланса.

    \item Генерируемые описания сочетали акцент на визуальных деталях с эмоциональным нарративом, создавая увлекательные истории, а не просто сухие отчеты о результатах.
\end{itemize}

Успех "{}Дазановых игрищ"{} продемонстрировал эффективность использования ИИ для создания богатого нарративного контента, способного вовлекать аудиторию и стимулировать творческое взаимодействие внутри сообщества. Проект также выявил потенциал ИИ в роли беспристрастного арбитра, способного генерировать результаты, воспринимаемые участниками как справедливые и увлекательные.

\subsubsection{Гибридные подходы в современных ВПИ}

Помимо специализированных проектов, ряд традиционных ВПИ начинает интегрировать элементы автоматизации для облегчения работы вердеров и расширения игровых возможностей:

\begin{itemize}
    \item \textbf{Автоматизированные экономические системы} — некоторые проекты, такие как "{}Эсенвальд"{} и "{}Империал"{}, внедряют алгоритмические решения для расчета экономических показателей, налоговых поступлений и демографических изменений~\cite{wpg-automation}.

    \item \textbf{Системы симуляции сражений} — проекты вроде "{}Тирания"{} используют специализированные калькуляторы и генераторы для моделирования исходов военных столкновений на основе численности войск, их качества, тактики и других факторов~\cite{battle-automation}.

    \item \textbf{Полуавтоматический вердинг} — в проектах с большим количеством игроков иногда применяются шаблоны и автоматизированные ответы для типовых приказов, что позволяет вердерам сосредоточиться на более сложных и нестандартных ситуациях~\cite{wpg-semiauto}.
\end{itemize}

\subsubsection{Экспериментальные подходы и перспективные направления}

Помимо уже реализованных проектов, в сообществе ВПИ обсуждаются и тестируются новые подходы к интеграции ИИ:

\begin{itemize}
    \item \textbf{Коллаборативное вердерство} — системы, где искусственный интеллект генерирует предварительные версии вердиктов, которые затем редактируются и утверждаются человеком-вердером. Такой подход позволяет сочетать креативность и эффективность ИИ с опытом и суждением человека.

    \item \textbf{Динамические игровые миры} — использование ИИ для непрерывной генерации событий, персонажей и сюжетных поворотов, создающих ощущение "{}живого"{} мира, реагирующего на действия игроков. Подобные системы позволяют существенно расширить глубину и вариативность игрового опыт.

    \item \textbf{Персонализированный игровой опыт} — адаптация контента, генерируемого ИИ, под предпочтения и стиль игры конкретных участников. Например, система может формировать вердикты с разным уровнем детализации и стилистическими особенностями в зависимости от предпочтений игрока.

    \item \textbf{Мультимодальные взаимодействия} — интеграция текстовых описаний с автоматически генерируемыми картами, изображениями и даже аудиоконтентом для создания более иммерсивного игрового опыта~\cite{catalogwpg2023}.
\end{itemize}

\subsubsection{Дискуссии в сообществе ВПИ о роли ИИ}

Внедрение искусственного интеллекта в военно-политические игры вызывает активные дискуссии в сообществе, где можно выделить несколько основных позиций:

\begin{itemize}
    \item \textbf{Сторонники технологического прогресса} считают, что ИИ способен решить фундаментальные проблемы жанра — нехватку вердеров, ограниченную скорость обработки приказов и субъективность в принятии решений. Они видят в автоматизации путь к возрождению и расширению сообщества ВПИ.

    \item \textbf{Традиционалисты} выражают опасения, что использование ИИ лишит игру человеческого творческого элемента, который они считают неотъемлемой частью жанра. По их мнению, субъективность и личный стиль вердера являются не недостатком, а уникальной особенностью ВПИ.

    \item \textbf{Прагматики} предлагают гибридные подходы, где ИИ используется для автоматизации рутинных аспектов вердинга, в то время как ключевые сюжетные решения и разрешение сложных ситуаций остаются прерогативой человека~\cite{wpg-ai-debate}.
\end{itemize}

Анализ существующих решений по автоматизации геймплея в ВПИ демонстрирует, что интеграция искусственного интеллекта в жанр находится на начальной стадии с акцентом на экспериментальные подходы и ограниченное применение в отдельных аспектах игрового процесса. Однако быстрое развитие технологий языковых моделей и растущий интерес сообщества создают предпосылки для более глубокой трансформации жанра в ближайшем будущем.
