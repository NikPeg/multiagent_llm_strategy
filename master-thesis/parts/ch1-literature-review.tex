В данной главе рассматриваются теоретические основы, необходимые для разработки мультиагентной текстовой стратегической игры на основе оркестрируемых языковых моделей. В первую очередь, представлен обзор жанра военно-политических игр (ВПИ), их история, ключевые особенности и механики, а также роль вердера (судьи) в организации игрового процесса. Далее рассматриваются принципы функционирования современных больших языковых моделей, их возможности и ограничения в контексте генерации игрового контента, оркестрации и создания мультиагентных систем. Особое внимание уделяется методам преодоления типичных проблем языковых моделей (галлюцинации, ограничения контекстного окна, согласованность генерируемого контента). Представлен анализ существующих подходов к применению ИИ в текстовых играх и, в частности, обзор существующих проектов ВПИ с использованием искусственного интеллекта, а также обсуждаются текущие дебаты в сообществе ВПИ относительно перспектив и этических аспектов использования ИИ в роли вердера. Глава завершается формулировкой ключевых требований и вызовов, которые необходимо учитывать при разработке автоматизированной системы проведения ВПИ.
\subsection{Определение и история военно-политических игр}

Военно-политические игры (ВПИ) представляют собой особый жанр текстовых стратегических игр, в которых игроки принимают на себя роль руководителей государств, политических фракций или других крупных организаций, взаимодействуя между собой и с игровым миром посредством текстовых приказов~\cite{wpg-glossary}. Ключевой особенностью ВПИ является наличие судьи (вердера), который интерпретирует приказы игроков и формирует вердикты — текстовые описания результатов действий.

В соответствии с определением, представленным в глоссарии ВПИ, \textit{«военно-политическая игра — это разновидность стратегической игры, в которой игроки управляют государствами или иными политическими образованиями, принимая решения относительно их внешней и внутренней политики, экономического развития, военных действий и других аспектов государственного управления»}~\cite{wpg-definition}.

Фундаментальная особенность ВПИ заключается в текстовом формате взаимодействия, который отличает этот жанр от коммерческих стратегических компьютерных игр. Игровой процесс ВПИ строится вокруг системы \textit{приказ-вердикт}, где приказ — это сформулированная игроком воля относительно действий управляемого объекта, а вердикт — письменный ответ вердера, описывающий результаты выполнения этого приказа~\cite{dtf-wpg-article}.

\subsubsection{Историческое развитие жанра}

История ВПИ имеет глубокие корни, восходя к традиционным военным играм и стратегическим симуляциям. Прародителями жанра можно считать настольные военные игры (варгеймы), появившиеся в XIX веке, в частности кригшпиль — немецкую настольную игру, использовавшуюся для подготовки офицеров~\cite{wpg-encyclopedia}. Кригшпиль впервые реализовал концепцию \textit{тумана войны} и стремился к реалистичной симуляции боевых действий на стратегическом уровне.

Дальнейшее развитие жанр получил с появлением настольной игры \textit{Дипломатия}, сочетавшей военные действия с элементами переговоров и политики. Особенно важным этапом для формирования ВПИ стали почтовые варианты игры Дипломатия (Play-by-mail), где игроки обменивались ходами в текстовом формате через обычную почту~\cite{wpg-history-article}.

С развитием интернета в 1990-х годах возникли первые онлайн-сообщества, практикующие текстовые стратегические игры на форумах и в чатах. Однако доступные исторические данные указывают, что полноценное формирование ВПИ в современном понимании на русскоязычном пространстве произошло в начале 2010-х годов с появлением соответствующих сообществ в социальной сети ВКонтакте~\cite{wpg-dtf}.

Одной из старейших и наиболее устойчивых ВПИ в русскоязычном сегменте является проект \textit{Империал}, основанный в 2013 году и функционирующий до настоящего времени. Другие значимые проекты включают \textit{Реальный Мир} (бывший ООН), \textit{Цивилизация} и \textit{Эсенвальд}~\cite{catalogwpg}.

\subsubsection{Жанровые разновидности ВПИ}

В процессе развития жанр ВПИ дифференцировался на несколько основных направлений:

\begin{itemize}
    \item \textbf{Классическая ВПИ} — проект, где игроки управляют государствами и имеют возможность действовать во всех сферах государственной политики, включая экономику, военное дело, дипломатию и социальную сферу~\cite{wpg-glossary}.

    \item \textbf{Командно-штабная игра (КШИ)} — разновидность ВПИ с акцентом на военной составляющей, где игроки делятся на команды, представляющие штабы противоборствующих армий~\cite{wpg-dtf}.

    \item \textbf{Виртуальное государство} — поджанр, фокусирующийся на детальной симуляции внутренней политики одного государства, где игроки могут управлять партиями, компаниями или высокопоставленными чиновниками~\cite{wpg-definition}.
\end{itemize}

Помимо этого, ВПИ классифицируются по сеттингу (реальная история, альтернативная история, фэнтези, научная фантастика и т.д.) и по игромеханике, которая может варьироваться от преимущественно \textit{ролеплейной} (текстовое взаимодействие с судьей) до \textit{калькуляторной} (с использованием численных переменных и формул)~\cite{wpg-encyclopedia}.

\subsubsection{Международные аналоги}

За пределами русскоязычного пространства существуют аналогичные форматы игр, хотя они часто организованы иначе. В англоязычном сегменте распространены \textit{геополитические симуляторы} на специализированных форумах и сайтах, а также проекты на платформе Reddit. Эти проекты часто сочетают элементы текстовых игр с более структурированными компьютерными интерфейсами, что отличает их от классических ВПИ в российском понимании~\cite{wpg-dtf}.

Таким образом, военно-политические игры представляют собой сложившийся жанр на стыке стратегии, ролевой игры и коллективного сторителлинга, имеющий богатую историю и устоявшиеся традиции. Этот формат продолжает привлекать участников благодаря уникальному сочетанию стратегической глубины, творческой свободы и социального взаимодействия.

\subsection{Специфика роли вердера в ВПИ и проблемы её автоматизации}

Вердер (судья) является центральной фигурой в военно-политических играх, выполняя функцию посредника между игроками и игровым миром. В отличие от настольных игр с жестко фиксированными правилами или компьютерных стратегий с заранее запрограммированными алгоритмами, ВПИ опираются на человеческое суждение для интерпретации игровых ситуаций и формирования нарративного опыта.

\subsubsection{Функции вердера в ВПИ}

Роль вердера в ВПИ многогранна и включает в себя следующие ключевые функции:

\begin{enumerate}
    \item \textbf{Интерпретация приказов} — вердер анализирует текстовые приказы игроков, оценивает их осуществимость в рамках игрового мира и правил, интерпретирует их намерения~\cite{rpg-gamemaster}.

    \item \textbf{Генерация вердиктов} — формирование связного текстового описания результатов выполнения приказов, включая как успешные, так и неудачные исходы, с соблюдением стилистического и смыслового единства игрового мира~\cite{wpg-glossary}.

    \item \textbf{Поддержание целостности игрового мира} — обеспечение согласованности происходящих событий, отслеживание изменений в игровой вселенной, контроль за соблюдением внутренней логики мира и технологического уровня эпохи~\cite{dtf-wpg-article}.

    \item \textbf{Арбитраж конфликтов} — разрешение спорных ситуаций между игроками, особенно в случае военных столкновений или дипломатических кризисов, с соблюдением баланса и справедливости~\cite{rpg-gamemaster}.

    \item \textbf{Развитие сюжета} — создание и внедрение игровых событий (ивентов), которые направляют развитие общего нарратива игры и создают новые возможности для взаимодействия игроков~\cite{narratology-games}.

    \item \textbf{Ведение учета игровых параметров} — отслеживание экономических, военных, дипломатических и других показателей стран, что необходимо для обеспечения последовательности и справедливости игрового процесса~\cite{wpg-dtf}.
\end{enumerate}

В традиционных ВПИ все эти функции выполняются человеком или группой людей, что требует значительных временных затрат, глубокого знания игрового мира и механик, а также определенных творческих способностей. Вердер должен быть одновременно объективным арбитром, который следит за соблюдением правил, и творческим рассказчиком, способным генерировать увлекательный нарратив.

\subsubsection{Формы организации вердерства}

В сообществе ВПИ сложились различные подходы к организации вердерской работы:

\begin{itemize}
    \item \textbf{Централизованное вердерство} — классическая модель, где один главный администратор или небольшая группа вердеров обрабатывает все приказы всех игроков. Это обеспечивает единство стиля и согласованность мира, но создает высокую нагрузку на администрацию~\cite{wpg-dtf}.

    \item \textbf{Распределенное вердерство} — система, где разные вердеры отвечают за различные аспекты игры (экономика, военное дело, дипломатия) или за определенные регионы игрового мира. Это снижает нагрузку, но требует тщательной координации для поддержания целостности мира~\cite{rpg-gamemaster}.

    \item \textbf{Парные вердерства} — инновационный подход, где игроки объединяются в пары и вердят приказы друг друга под общим контролем администрации. Согласно описанию механики парных вердерств от каталога ВПИ, такая система позволяет решить проблему нехватки вердеров и расширить административный ресурс проекта~\cite{paired-verdership}.
\end{itemize}

Несмотря на разнообразие подходов, все формы организации вердерства сталкиваются с общими проблемами: высокой трудоемкостью, субъективностью оценок, риском несогласованности игрового мира при участии нескольких вердеров и ограниченной скоростью обработки приказов.

\subsubsection{Вызовы при автоматизации роли вердера}

Автоматизация роли вердера с использованием искусственного интеллекта, в частности языковых моделей, представляет собой сложную междисциплинарную задачу, сопряженную с рядом специфических вызовов:

\begin{enumerate}
    \item \textbf{Понимание контекста и намерений} — языковая модель должна корректно интерпретировать приказы игроков, которые могут быть неоднозначными, содержать импликации или опираться на предыдущие события в игре~\cite{ai-narrative}.

    \item \textbf{Поддержание долгосрочной согласованности} — одной из ключевых сложностей является необходимость поддерживать целостность игрового мира на протяжении длительных сессий, отслеживая множество параметров и событий~\cite{ai-game-simulation}.

    \item \textbf{Баланс между следованием правилам и творческой свободой} — автоматизированная система должна одновременно придерживаться установленных правил игры и генерировать интересный, разнообразный контент, что требует тонкого баланса между структурированностью и креативностью~\cite{ai-narrative}.

    \item \textbf{Справедливость и непредвзятость} — алгоритмическая система должна избегать фаворитизма и обеспечивать справедливое отношение ко всем игрокам, что особенно важно в конфликтных ситуациях~\cite{rpg-gamemaster}.

    \item \textbf{Адаптация к изменяющимся условиям} — по мере развития игры модель должна адаптироваться к новым технологическим уровням, изменениям в политическом ландшафте и другим динамическим аспектам игрового мира~\cite{narratology-games}.

    \item \textbf{Ограничения контекстного окна} — современные языковые модели имеют ограниченное контекстное окно, что создает сложности при необходимости учитывать долгую историю игры~\cite{llm-limitations}.

    \item \textbf{Генерация галлюцинаций} — склонность языковых моделей к генерации фактически неверной информации представляет серьезную проблему для поддержания согласованности игрового мира~\cite{llm-hallucinations}.
\end{enumerate}

\subsubsection{Потенциальные преимущества автоматизации вердерства}

Несмотря на значительные технические и методологические вызовы, автоматизация роли вердера потенциально предлагает ряд существенных преимуществ для жанра ВПИ:

\begin{itemize}
    \item \textbf{Масштабируемость} — автоматизированная система способна обрабатывать значительно большее количество приказов в единицу времени по сравнению с человеком-вердером, что потенциально позволяет увеличить число игроков и динамику игры~\cite{ai-game-simulation}.

    \item \textbf{Доступность} — снижение зависимости от человеческих ресурсов делает формат ВПИ более доступным, позволяя организовывать игры без необходимости привлечения большого числа администраторов~\cite{ai-narrative}.

    \item \textbf{Последовательность} — правильно настроенная система может обеспечивать более последовательное применение правил по сравнению с человеком, который подвержен влиянию настроения, усталости и личных предпочтений~\cite{llm-applications}.

    \item \textbf{Инновационность нарратива} — языковые модели, обученные на огромных корпусах текстов, могут генерировать неожиданные сюжетные повороты и ситуации, обогащая игровой опыт~\cite{ai-narrative}.

    \item \textbf{Аналитические возможности} — автоматизированная система может отслеживать и анализировать сложные взаимосвязи между различными аспектами игрового мира, обеспечивая более глубокую симуляцию~\cite{ai-game-simulation}.
\end{itemize}

\subsubsection{Подходы к решению проблем автоматизации}

Для преодоления описанных вызовов при автоматизации роли вердера могут применяться различные стратегии:

\begin{itemize}
    \item \textbf{Мультиагентный подход} — разделение функций вердера между несколькими специализированными агентами, где каждый отвечает за определенный аспект игры (экономика, военное дело, дипломатия)~\cite{multi-agent-systems}.

    \item \textbf{Дополнение генерации извлечением (RAG)} — использование дополнительной информационной базы, из которой модель может извлекать фактическую информацию о текущем состоянии игрового мира, что снижает риск галлюцинаций~\cite{rag-applications}.

    \item \textbf{Человеческий надзор} — сохранение роли человека-модератора, который может вмешиваться в критических ситуациях, корректировать курс игры и разрешать сложные конфликты~\cite{human-ai-collaboration}.

    \item \textbf{Структурированные протоколы взаимодействия} — разработка четких форматов для подачи приказов и генерации вердиктов, что облегчает задачу интерпретации для языковой модели~\cite{llm-applications}.

    \item \textbf{Инкрементальное обновление контекста} — разработка механизмов для обновления и сжатия контекстной информации, что позволяет преодолеть ограничения контекстного окна языковых моделей~\cite{llm-context-window}.
\end{itemize}

Таким образом, специфика роли вердера в ВПИ представляет собой уникальный комплекс задач, требующий как технических решений в области искусственного интеллекта, так и глубокого понимания принципов нарративного дизайна и игровой механики. Автоматизация этой роли, хотя и сопряжена со значительными вызовами, открывает новые горизонты для развития жанра ВПИ, потенциально делая его более доступным, динамичным и масштабируемым.
