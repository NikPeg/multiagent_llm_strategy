\subsection{Опрос потенциального пользователя А.}
\textbf{Я:}\\
Продолжаю серию вопросов про релоад.\\

\textbf{А:}\\
Вопрошай.\\

\textbf{Я:}\\
Тебе было бы удобнее играть в ВКонтакте или в Telegram?\\

\textbf{А:}\\
В Telegram.\\

\textbf{Я:}\\
\textit{(удивление)}\\
Почему так?\\

\textbf{А:}\\
Разметка.\\

\textbf{А:}\\
Возможно, релоад уже не стоит проводить,\\
ожидания перегреты.\\

\textbf{Я:}\\
Да, в перспективе на каждого персонажа по ассистенту\\
Что такое OpenAI агент?\\

\textbf{А:}\\
Пусть они друг в друга через некий буфер кидают информацию\\
Они вроде хотели сделать систему персонализированных ассистентов\\

\textbf{Я:}\\
Вообще не гуглится\\

\textbf{А:}\\
Странно, я помню новости\\
Возможно, это было в другом мире\\

\textbf{Я:}\\
Ааа OpenAI это openai\\
Тогда да\\

\textbf{Я:}\\
Короче, буду валидировать с тобой свои мысли по поводу релоада\\
Самой большой сложностью в этой задаче я вижу как раз размер контекстного окна\\
Например, если Никите в 7 классе было нормально держать в голове ситуацию в 15 странах одновременно, то для GPT это пока нереально\\
Поэтому думаю сделать оркестр ассистентов, которые будут переговариваться друг с другом\\
Один ассистент на всю игру (около ста лет), в которого будут закидываться только краткие сводки за последние 10 лет\\
Один отвечающий за десятилетие\\
Один за год (то есть за игровой день)\\
Еще есть сложность во взаимодействии ассистентов. Нужно как-то мотивировать игроков переговариваться друг с другом, устраивать всякие саммиты и отправлять послов. Но это, возможно, на первом этапе можно и руками делать, все равно я в ручном режиме буду все контролировать\\

\textbf{А:}\\
Насколько я помню, ты буквально хотел, чтобы советники и прочие персонажи были отдельными личностями\\
Звучит как очевидная, максимально учебниковая задача для OpenAI Агентов\\

\textbf{А:}\\
Нет\\
Донат разрушает геймплей, по моему мнению\\

\textbf{Я:}\\
А если они не будут влиять на твое преимущество? Например, не чипы информационного коммунизма, а скин?\\

\textbf{А:}\\
Э... Что?\\

\textbf{Я:}\\
Ну типа что-нибудь мелкое, но приятное, созданное скорее с целью поддержать разработчиков. Как цвет оружия в CS\\

\textbf{А:}\\
Это же текстовая RPG\\
Я могу буквально делать что хочу\\
В этом и прелесть\\

\textbf{Я:}\\
Ну типа что-нибудь мелкое, но приятное, созданное с целью поддержать разработчиков...\\

\textbf{А:}\\
Сложно сказать\\

\textbf{Я:}\\
Таки релоад и не будет в том виде, в каком я его планировал\\
То, как я сейчас вижу релоад — это скорее MVP\\
Вопрос третий, был бы ты готов платить за релоад? Сколько рублей в неделю?\\

\textbf{А:}\\
Если он будет дарить мне те самые эмоции от игры во Внеземную — полагаю, около 2000\\

\textbf{Я:}\\
В неделю? Раньше ты вроде говорил про 2000 в месяц\\

\textbf{А:}\\
Да, в неделю. Инфляция :)\\

\textbf{Я:}\\
Чудеса\\

\textbf{Я:}\\
Ты бы предпочел, чтобы я нанял Водяна и вердикт выносил он?\\

\textbf{А:}\\
Я не знаю\\

\textbf{Я:}\\
Ладно, я тебя понял\\
Не знаю, мне хотелось бы, чтобы в релоад могли поиграть много людей, хотя бы 20 человек\\

\textbf{А:}\\
Я не позволю Водяну прикасаться к Инферии как судье\\
Так что лучше GPT\\

\textbf{Я:}\\
Безальтернативный выбор сработал\\
Хотя почему безальтернативный\\
У тебя всё ещё есть возможность нанять меня как судью за 50к в месяц!\\

\textbf{Я:}\\
Очевидно, я еще не придумал как монетизировать, но можно же как-нибудь\\

\textbf{А:}\\
Сейчас ответ нет, но возможно ты придумаешь что-то, с чего я удивлюсь и скажу да\\
Не знаю, возможность нанять кого-то, кто будет рассказывать тебе полный лор Инферии, переодевшись в персонажа из Викии Третьей\\

\textbf{Я:}\\
Ха-ха)))\\
Окей, следующий вопрос\\
Насколько, по твоим оценкам, игра, использующая GPT, принесет тебе меньше эмоций?\\
Допустим, не полный GPT, а я буду использовать его для генерации текста или картинок\\

\textbf{А:}\\
Это не проблема само по себе, особенно если будет судья-человек. Я опасаюсь GPT из-за того, что она тяготеет к среднему, а Инферия — это буквально государство-экстремум.\\
И она может\\
НЕ ПОНЯТЬ ИДЕЮ\\
УПУСТИТЬ СУТЬ\\

\textbf{Я:}\\
Насколько тебе нравится рандом в ВПИшках? Хотел бы ты, чтобы он был в релоаде? Как сильно влиял бы?\\

\textbf{А:}\\
Не нравится. Не хотел бы. Влиял на косметику, типа "Небо сегодня было пасмурным, Викия грустно смотрела в окно". Псевдослучайные штуки, типа катаклизмов — ок.\\

\textbf{Я:}\\
\textit{абстрактные рассуждения}\\
Не знаю, насколько ты тут прав. Когда мы делаем что-то, не существовавшее раньше, очень сложно рассчитать вероятность успеха. Условно, когда кто-то вкладывает деньги в проект, он вряд ли может точно оценить, получится ли он вообще.\
Поэтому я думаю сделать что-нибудь типа рандома с нормальным распределением, на параметры которого можно будет влиять в процессе игры. Например, вкладываешься в навык, получаешь сдвиг колокола вправо. Значит, вероятность прорывных открытий будет больше (но всё ещё не 100%).\\

\textbf{Я:}\\
Короче, я как-то пытаюсь добавить субъективность в игру, потому что GPT ей не обладает. Пока все мои эксперименты — это примерно так:\\
– разработать телепорт\\
– спустя пять лет упорного труда ваши ученые разработали телепорт!\\

\textbf{А:}\\
Да, однако это всё-таки факторная штука.\\
То есть, Зимбабве не разработает нечто подобное\\
Вообще\\
Какая бы удача не случилась\\

\textbf{А:}\\
Да, звучит отлично\\

\textbf{А:}\\
Кстати, это может интересно работать с разными отраслями\
В смысле, колокол же может не только двигаться вправо и влево\\
Но и становиться более или менее дисперсным\
Полагаю, обильное финансирование "науки в целом" — это про утолщение правого хвоста.\\
То есть, про некую систему, улавливающую таланты и раскрывающую их в разных сферах\
А вот финансирование системы образования, полагаю, будет именно что двигать колокол, потому что талантов будет становиться больше в среднем\\

\textbf{Я:} Как ты относишься к генерации картинок в ВПИ?\\

\textbf{А:}\\
Идея сомнительная.\\
Тебе надо или мастерски владеть генерацией картинок,\\
чтобы на сотне изображений персонаж выглядел одинаково, и ситуация учитывалась\\
Либо лучше не надо.\\

\textbf{Я:}\\
А кто сказал, что везде будет один и тот же персонаж?\\
Например, там текст про строительство завода и там рабочие\\
Или митинг какой-нибудь\\

\textbf{А:}\\
Здания Инферии, вероятно, будут сильно отличаться от зданий другого места\\
Причем они будут в чём-то похожи друг на друга\\
При этом всё ещё функциональны\\
А вообще\\
Можно просто сделать эту функцию отключаемой\\

\textbf{Я}\\
Для тебя думаю сделать максимально настраиваемую игру\\
Чтобы ты, например, мог вообще отключить GPT при желании\\
Кстати, есть идея добавить расшифровку голосовых сообщений. Чтобы можно было не писать вручную долгие сообщения\\
Ты бы пользовался?

\textbf{А}\\
Ты же знаешь, что у яндекс.клавиатуры отличный голосовой ввод?\\
Но вообще да, я бы пользовался.\\

\textbf{Я}\\
Нет

\textbf{Я}\\
Было бы тебе интересно в релоаде смотреть на графики?\\
Например, курс джибриллита к митолу или диаграмма доходов/расходов Инферии

\textbf{А}\\
ДА\\
Мой движок для ВПИ, который я все думал создать\\
Должен был базироваться как раз на графиках

\textbf{Я}\\
Скажи, какие ты показатели хочешь видеть на графиках? Пока думаю ВВП, казна, численность населения, может что-то ещё?

\textbf{А}\\
Зависит от того, насколько ты готов в это углубляться.\\
К тому же, не забывай, что Инферия — это плановое государство, у него могут быть разные странные графики.\\
Но если бы мне в реальном времени показывали АК-модель...\
Проблема в том, что она рыночная.\\
А мы не экономисты, чтобы адаптировать сложную модель так сразу.\\
Так что может быть достаточно и графика остатка Солоу,\\
такого как "Общая факторная производительность", или "Мера нашего незнания", также известная как "Коэффициент прогресса".\\
Грубо говоря, это чтобы я мог отслеживать степень инновационности экономики.\\
Поскольку, как известно, остаток Солоу выражает инновационность экономики. И только в СССР из всех сверхдержав он был отрицательным.\\

\textbf{Я}\\
Как ты относишься к рамкам эпохи?\\
Я помню, что плохо, но обоснуй

\textbf{А}\\
Прогресс — это эндогенное явление общества.\\
Проблема долгих эпох прошлого — в их эволюционности развития.\\
Сделать что-то в феодализме очень сложно.\\
Микроулучшения накапливаются столетиями, и этим улучшениям активно противодействуют.\\
Фактически, существует ряд условий, в которых эта ситуация переламывается.\\
И если эти условия возникают, то она изменяется.\\
А если не возникают, то она может сохраняться тысячелетиями.\\
Из-за чего идея "Сейчас эпоха X, вы не можете перейти в эпоху Y в ближайшие n лет, потому что никак" плоха.\\
Мы играем в ВПИ, чтобы желать: "А вот если бы у Петра Первого был учебник по макроэкономике..."

\textbf{Я}\\
Ну вот кажется, я такое не хочу.\\
То есть если ты в начале игры скажешь "разработать ядерку", тебе ответят "что такое ядерка".

\textbf{А}\\
Впрочем, Золотой Полувек — это уже не та эпоха, где ты хочешь быстрее пройти этот период с мечами.\\
Это же период средней индустриализации. Самое интересное.

\textbf{Я}\\
Короче, пока я думаю так: ограничения эпохи будут, но я их буду активно менять. Допустим, если твои ученые изобретут электричество (а тут, как ты помнишь, влияет рандом), то это уже начало следующей эпохи.

\textbf{А}\\
Стоп, погоди.\\
Где мы стартуем?

\textbf{Я}\\
Либо GPT ошибётся, и в самом начале игры пришлет ивент типа "ваши ученые изобрели ядерку".\\
Конец Золотого Полувека.

\textbf{А}\\
Если помнишь, Алинэхо получила своё крыло именно за разработки в электричестве.

\textbf{Я}\\
(Изображение с пометкой "БЛИН")

\textbf{А}\\
То есть, это уже после того, как инферцы запустили кек-ракету на механике.

\textbf{Я}\\
Интересно, успел ли GPT обучиться на твоих книгах.

\textbf{А}\\
Причем, на десятилетия после.

\textbf{Я}\\
То есть ты бы хотел, чтобы я следовал лору ДПвС?

\textbf{А}\\
Ну, даже если отбросить ДПвС и взять чистый лор Инферии, то кек-ракета уже была.

\subsection{Опрос потенциального пользователя Т.}\\
\textbf{Т}\\
Рассудительная боевка — хорошо.\\
Чем больше интерактива, тем веселее.

\textbf{Я}\\
Что такое рассудительная?\\
Типа как в шахматах?

\textbf{Т}\\
То, что ты Алёне писал.\\
С попыткой описать тактику и стратегию в ключевые моменты.\\
Война — дело долгое, а потому наполненное событиями.

\textbf{Я}\\
У меня пока ноль идей, как это с GPT-ассистентами реализовать. Но буду думать.\\
Короче, ты бы выбрал боевку?

\textbf{Т}\\
Да\\
Как много вердов ты бы хотел? ВЕРД — "вердикт" от админа. Например, ты пишешь "построить золотой мост". Админ отвечает: "у вас нет золота". Это верд.

\textbf{Т}\\
А второй можно, если ты отмел этот?

\textbf{Я}\\
Несколько раз в день/раз в день/раз в несколько дней/чаще?

\textbf{Т}\\
А второй можно, если ты отмел этот?\\

\textbf{Я}\\
Как понять отмел?

\textbf{T}\\
Ну вот ты сказал, что золота нет — я сижу и грущу до следующего или могу передумать?\\
Или это последствия?

\textbf{Я}\\
Понял тебя.\\
Ну вот это еще один вопрос к разработке, надо подумать.

\textbf{Т}\\
«Как много вердов ты бы хотел? ВЕРД — "вердикт" от админа. Например, ты пишешь "построить золотой мост". Админ отвечает: "его забрала Алёна". Это верд»\\
Так правильнее. 🔥🦔

\textbf{Я}\\
В идеале я бы сделал, что игрок пишет сколько угодно приказов, и раз в сутки GPT ему отвечает на все. Но возможно, это будет слишком дорого.\\
В связи с этим следующий вопрос — сколько бы ты был готов платить за участие в игре?

\textbf{Я}\\
В идеале я бы сделал, что игрок пишет сколько угодно...\\

\textbf{Т}\\
Так слишком много будет действий?\\

\textbf{Я}\\
В чем минусы? Это же не "игра по шагам", здесь нет ходов. В этом и плюс, что в ВПИ ты можешь писать сколько хочешь, и про что хочешь. Полная свобода.

\textbf{Я}\\
В связи с этим следующий вопрос — сколько бы ты бы...\\
300 в неделю нормально, даже 500 нормально\\
После 1000 перебор, наверное. 👍🎅

\textbf{Я}\\
Тогда вопрос, это хорошо или плохо?

\textbf{Т}\\
Надо будет тогда самому себя контролировать, чтобы не развить летающих робо-котов за день.

\textbf{Я}\\
Не получится :) (надеюсь, у меня удастся ограничить GPT, чтобы он такого не допускал).

\textbf{Т}\\
И все их за день написать в GPT

\textbf{Т}\\
Так нормально

\textbf{Я}\\
Все равно будут рамки эпохи, если мы живем в стимпанке, то за год ты электронный компьютер не соберешь. 👍👨‍💻\
Пока я планирую сделать это так: на каждое действие под капотом GPT будет оценивать, как много времени это займет. И ставить таймер\\
1 день = 1 год

Допустим, ты пишешь приказ: начать разработку квантовых компьютеров. GPT думает, ага, чтобы в стимпанке создать квантовый компьютер, нужно 150 лет. Тогда результаты твоего исследования придут только через 150 дней.\
Может найду более простой вариант, и сделаю это как-то по-другому!\\

\textbf{Т}\\
Звучит сложно по промптам\\
Но готов помочь\\

\textbf{Я}\\
Да, тут основная сложность проекта в промптах

\textbf{Т}\\
У меня есть опыт

\textbf{Я}\\
Как ты относишься к рандому в играх? Он тебя скорее бесит из-за дисбаланса или нравится?\\

\textbf{Т}\\
От количества зависит\\
Он конечно нужен, но в соотношении 7:3 где 7 это действия игроков\\
(Очень примерно)\\

\textbf{Я}\\
Понятно\\
А вот если игрок не играет\\
Допустим, завал на работе, и нет времени на игру\\
Что делать?

\textbf{Т}\\
Ну смотря сколько времени он будет занят?

Если один день не отвечает, то и ладно, застой в экономике, бывает\
Если неделю, то грустно\\

\textbf{Я}\\
Услышал тебя\\
Ну кстати здесь твое мнение прям сильно расходится с большинством

\textbf{Т}\\
А у остальных какое?

\textbf{Я}\\
Собственно основной претензией Гоги ко мне за Внеземную был факт, что у него полстраны захватили, пока он был АФК

\textbf{Т}\\
Хах

\textbf{Я}\\
И он бы хотел, чтобы за него играл ИИ в это время

\textbf{Т}\\
Ну я и говорю, что один день можно простой сделать\\
Потом надо или ждать или как-то еще решать

\textbf{Я}\\
Возможно

\textbf{Т}\\
Это сложно, потому что он может сильно накосячить\
Вообще, я для себя не вижу проблемы выделить 15 минут прочитать что произошло и написать пару действий)\\
Но это мое отношение\\
Если подписался играть, то уж будь добр)\\

\textbf{Я}\\
Не знаю насчет 15 минут\
Если будет много дискуссий в беседке, то кто-то может захотеть их тоже читать\
Но да, на первом уровне как будто 15 минут в день нормально\\

\textbf{Т}\\
Ну я ж не играл

\textbf{Я}\\
Как тебе было бы интереснее общаться с другими игроками? Внутри бота или в беседке?\
Я имею в виду, можно в боте сделать функцию "отправить телеграмму другому правителю"\\
И писать через нее\
А можно самому пойти в личку/в беседку и обсудить с ним вопрос\\

\textbf{Т}\\
Скорее так интереснее\
Я за общение)\\

\textbf{Я}\\
Было бы тебе удобно использовать голосовые сообщения для написания приказов?

\textbf{Т}\\
Скорее нет)\\

\textbf{Я}\\
Хорошо\\

\textbf{Т}\\
Я просто думаю, что не успею подумать за время голосового\\
Поэтому проще писать\\

\textbf{Я}\\
Понимаю\
Хотел бы ты видеть картинки/графики в игре?\\

\textbf{Т}\\
Да, думаю будет интересно)

\textbf{Я}\\
Допустим, диаграмма расходов/доходов твоего государства

\textbf{Т}\\
Я больше подумал про иллюстрации к действиям\\
Я скорее всего на диаграммы меньше внимания обращу, ибо я не очень разбираюсь и мне они мало что дадут)\\

\textbf{Я}\\
Понятно\\
Иллюстрации к вердам/новостям тоже отличная идея\\

\textbf{Т}\\
Именно

\textbf{Я}\\
Как ты думаешь, что в большей степени будет влиять на твои эмоции во время игры?\
Например, процесс обдумывания/написание приказов/чтение вердов/общение с другими игроками/просмотр картинок/что-то еще?\\

\textbf{Т}\\
Я очень надеюсь, что получится соблюсти баланс между интересными ситуациями и тем, чтобы игровой процесс имел смысл)\
Мне, конечно, будет интереснее всего продумывать множество вариантов гениальных планов по развитию и захвату власти и влияния\
Но в целом все вышеперечисленное меня радует\\

\textbf{Я}\\
Понял, отлично\\
А как ты относишься к исходному дисбалансу? Например, у всех разные территории/доходы/население\\

\textbf{Т}\\
В пределах разумного, но в рамках сеттинга так и должно быть же)\\

\subsection{Опрос потенциального пользователя Ж.}\\
\textbf{Я}\\
Тебе было бы удобнее играть в ВКонтакте или в Telegram?

\textbf{Ж}\\
Telegram

\textbf{Я}\\
Как ты относишься к использованию GPT в ВПИ? Как думаешь, это бы ухудшило или улучшило твои эмоции от игры?

\textbf{Ж}\\
Сложно сказать, может случиться так, что и мастер и игрок - боты

\textbf{Я}\\
Да, будет интересно

\textbf{Ж}\\
В этом есть плюсы и минусы\\
Нельзя сказать, что это точно плохая идея, но нельзя сказать и обратного

\textbf{Я}\\
Какие ты видишь минусы?

\textbf{Ж}\\
Минусы могут быть в том, что бот может в какой-то момент вести 2 разные ветки событий для разных игроков\\
Но я не уверен в этом\\
Смотря насколько глубоко будет внедрен бот

И я не знаю, как ты все настроил\\
Возможно, будет отлично

\textbf{Я}\\
Тоже об этом думаю. Скорее всего, будет какой-нибудь оркестр ассистентов. Например, один для Инферии, другой для ХФ, третий для взаимодействия между ними\\
Пока никак не настроил, вот провожу исследование!

\textbf{Ж}\\
Это твой диплом?

\textbf{Я}\\
Нет

\textbf{Ж}\\
Или курсовая?\\
Понятно

\textbf{Я}\\
Исключительно бизнес-проект\\
Сколько бы ты был готов платить в неделю?

\textbf{Ж}\\
Понятно\\
У Инферии опять будут чипы в каменном веке

\textbf{Я}\\
Скорее всего, будет три варианта игры\\
Бесплатная, с низким уровнем погружения\\
Средняя, с более детальным\\
Высокая, для А\\
Разница в количестве вердов в день

\textbf{Ж}\\
Я бы попробовал бесплатный, потом например на пару дней средний, если понравится, то остался бы на среднем\\
Нужно явно показать разницу между средним и бесплатным\\
Сколько готов платить?\\
Цена за бота меня устраивала

\textbf{Я}\\
Пока не продумывал точно, но, например 1 верд в день или 5

\textbf{Ж}\\
Но то в месяц было

\textbf{Я}\\
То есть 500 рублей в месяц?

\textbf{Ж}\\
Примерно так\\
Пока что точно не знаю

\textbf{Я}\\
Хорошо! Как ты относишься к рандому в ВПИ?

\textbf{Ж}\\
Думаю, что он нужен, но его должно быть не очень много

Temperature = 0.5

\textbf{Я}\\
Интересно\\
А какие ты бы хотел боевки в ВПИ?\\
Хотел бы участвовать в войнах сам или приказывать генералам?\\

\textbf{Ж}\\
Генералы - ИИ или игроки?

\textbf{Я}\\
Пока не знаю, будет ли именно проработка по личностям, но скорее ИИ

\textbf{Ж}\\
Я думаю, что и так и так было бы интересно

Ты можешь отправить генерала в бой\
А можешь и сам возглавить войско\\

\textbf{Я}\\
Что бы мотивировало тебя общаться с другими игроками? Например, не просто отправить послов, а написать в беседку/личку

\textbf{Ж}\\
Думаю, что обсуждение условий союзов, контрактов и войны\
Но это лучше сделать через какой-то чат с ботом, который будет все фиксировать\\

\textbf{Я}\\
Почему так? В личку неудобно?\\

\textbf{Ж}\\
Это можно будет добавить в общий контекст\
И в случае чего с этим связать например утечку\
Если к тебе кто-то шпионов отправил или подобное\
Это как один из вариантов, почему так лучше\\

\textbf{Я}\\
Понятно, хорошо

\subsection{Опрос потенциального пользователя Г}\\
\textbf{Я}\\
Привет! Планирую в следующем году запускать релоад.\\
Игра будет идти примерно полгода. Будешь участвовать?\\
Хочу задать несколько вопросов по ожиданиям 🔥🎅

\textbf{Г}\\
Привет, а если буду задерживаться с ответами, это тебе может испортить администрирование? Или "если вы отсутствовали пару дней и вам за это время нанесли ядерный удар - ваши проблемы"?

\textbf{Я}\\
Это проблема, которую мне нужно заранее учесть. Скорее всего, события в твоей стране будут происходить и без твоего участия, ибо это большая система с множеством людей. Тебе бы такое не нравилось?

\textbf{Г}\\
Я думаю, что из двух зол это меньшее (т.е. я бы предпочел так). Поиграть и пофантазировать - с удовольствием, хочется. Но не могу гарантировать, что не будет какого-то завала по работе, когда мне не очень захочется ещё думать "а как бы сходить получше", и не хочется чтобы приходилось меня отмечать и ждать, когда я там отвечу. Поэтому, если такая ситуация будет, я пойму и приму её\\

\textbf{Я}\\
В прошлом подходе к созданию релоада, я собирался ввести три режима игры, различающиеся по уровню проработки. Условно, верды раз в день, раз в час и раз в минуту.\\
Кажется, если бы такое реализовать, твоя проблема была бы решена? (например, пусть А всё детально прорабатывает, а ты просто когда удобно большими мазками направляешь Лурк)\\

\textbf{Г}\\
Да, звучит достаточно разумно. Я бы, правда, сказал 2-3 раза в день, но это, наверное, адаптивная задача\\

\textbf{Я}\\
Хорошо. И все-таки, как бы ты отреагировал, если в твое отсутствие тебя атаковали ядерным оружием?\\

\textbf{Г}\\
Я думаю, что идеального варианта тут не получить, но в целом, если у моей страны нет никакого ПРО и всё такое, довольно странно если я пишу "строим ПРО и защищаемся", оно же не должно вот так вырастать из ниоткуда, когда нужно. То есть если я заранее не позаботился об этом - это моя проблема, и я бы не был недоволен\
При этом, конечно, если оно есть, оно должно работать без моей команды, мне не надо писать "используйте ПРО", понятно, что его надо использовать\
И выглядело бы довольно странно, если у нас вечером было средневековье, я весь день не отвечал, а всякие инферцы за день со средневековья прыгнули на ядерное оружие и все разбомбили, но это воспринимается как какая-то аномальная нереальная ситуация\
Так что, скорее нормально восприму. Я достаточно доверяю многоуважаемому администратору, чтобы понимать, что если такое произошло - в этом только моя вина 😁🙏

\textbf{Я}\\
Понятно! Тебе было бы удобнее играть в Telegram или ВКонтакте?\\

\textbf{Г}\\
Примерно одинаково, спокойно и там, и там буду\
Но чуть удобнее, наверное, было бы в Telegram, я там вероятнее уведомление увижу\\

\textbf{Я}\\
Хорошо. Как ты относишься к использованию GPT в ВПИ?\\
Как думаешь, насколько сильно это скажется на качестве вердов?\\

\textbf{Г}\\
Скорее, положительно, но:\\
лучше хотя бы прочитать, что он сгенерировал) (в частности, думаю, важна проверка контекста - языковая модель должна помнить, что у меня есть ПРО! Вот это, наверное, самый неприятный вид опыта, когда судья не учёл какую-то такую деталь, и оспаривать сложно, потому что все играют из той позиции, что тебя уже атаковали ядерным оружием)\\
не хочется, чтобы совсем уж без великого администраторского креатива) Но даже если условное 100\% GPT, было бы любопытно попробовать и так

\textbf{Я}\\
Отлично. В связи с этим возникает проблема, что GPT4o платный. И как оказалось, дорогой, Женя за месяц использования моего бота потратил ~1500 рублей.\
В связи с этим вопрос. Был бы ты готов платить за игру? Сколько рублей в неделю?\\

\textbf{Г}\\
Думаю, 300-500 рублей в месяц\\

\textbf{Я}\\
Думаю сделать три режима игры. Бесплатный, средний и дорогой. И в зависимости от режима разное ограничение на число вердов в день\
Кажется, вовлеченность человека в игру коррелирует с числом денег, которые он готов платить\
Насколько тебе нравится рандом в ВПИ? Хотел бы ты, чтобы он был? Или все как во Внеземной, все по решению администратора?\\

\textbf{Г}\\
Сложно сказать. Но, думаю, будет хорошо, если хоть какое-то количество рандома будет, так будет потенциально меньше априори недовольств решениями администратора\\

\textbf{Я}\\
Понимаю\\
Есть идеи, как реализовать боевку?\\
Для меня боевка во Внеземной и так была идеальной. Путин же не ездит на фронт, он только отдает приказы генералам\
Но видимо игрокам хочется какого-то действия\\
Ты бы предпочел, чтобы бои за тебя вел ИИ?\\

\textbf{Г}\\
Если на неформальном уровне, то можно даже\
Именно "сражаться", думаю, да\
Было бы интересно, на мой взгляд, участие в боевке, например, на таком уровне, что я знаю, что у меня очень лесистая страна, и я там даю тактико-стратегическое распоряжение загонять врагов вглубь леса, устраивать ловушки, размещать стрелков на деревьях, все дела\
Или, например, если я придумаю рыть вокруг замков ров и заполнять его голодными свиньями\\

\textbf{Я}\\
Интересно))

\textbf{Г}\\
То есть это такой же верд, только относящийся к "военной" сфере, а не прямо управление боевыми действиями

\textbf{Я}\\
Отлично, хотя пока не знаю, как это реализовать с GPT

В общем спасибо, если будут идеи, пиши! 👌

\subsection{Опрос потенциального пользователя В}\\
\textbf{Я}\\
Тебе было бы удобнее играть в ВКонтакте или в Telegram?

\textbf{В}\\
Без разницы

\textbf{Я}\\
А как много вердов ожидаешь?

Например, несколько раз в день, раз в день, раз в неделю

\textbf{В}\\
Поскольку ты гений я рад даже маленькому вердику

\textbf{Я}\\
Верд: ты гигачад

\textbf{Я}\\
Как ты относишься к GPT для написания вердов?

\textbf{В}\\
Сомневаюсь что ты сможешь её настроить так, чтобы она выдавала что-то на твоём уровне

\textbf{Я}\\
Посмотрим

Короче ваша армия разбита

\textbf{В}\\
Она несомненно сможет сделать красивый текст в ответе или что-то подобное

\textbf{Я}\\
Как ты относишься к рандому в ВПИ/ролках?

\textbf{В}\\
Но это будет как все продолжения Сомнабулы

\textbf{Я}\\
А что там в продолжении Сомнабулы?

\textbf{В}\\
Положительно

\textbf{Я}\\
Как бы тебе было удобнее общаться с другими игроками? В беседке или через бота?\\
(через бота - можно сделать в боте кнопку "написать письмо другому правителю" и отправлять напрямую другому игроку)\\

\textbf{В}\\
Через бота\\
И желательно абсолютная анонимность игроков\
Потому что иначе метаигра будет\\

\textbf{Я}\\
Чтобы не приехали в Казахстан выяснять отношения\\

\textbf{В}\\
Действительно

\textbf{Я}\\
В чем минусы метаигры?\\

\textbf{В}\\
Скучно и не соответствует духу ролевой игры

\textbf{Я}\\
Например, ТСГЕМ во Внеземной был организован в личке\
Я сам постфактум узнал, что он появился\
По-моему было интересно\\

\textbf{В}\\
Он именно в личке\\

\textbf{Я}\\
Но хорошо, значит, сделаем бота с анонимностью!\\

\textbf{В}\\
Это как личные переговоры\\

\textbf{Я}\\
Понимаю

\textbf{В}\\
А общий чат именно начинает метаигру и погружение портит\\

\textbf{Я}\\
Как бы ты хотел участвовать в войнах? Просто отдавать приказы генералам или сам расставлять силы, вести армию вперед?

\textbf{В}\\
От системы зависит, мне кажется второй вариант будет тебе сложнее рассчитывать\
Либо это будет странная версия камень-ножницы-бумага\
Делай тут как тебе самому удобно проводить\\

\textbf{Я}\\
Да пофиг, сделаем

Был бы ты готов платить за участие в релоаде? Сколько рублей в неделю?

\textbf{В}\\
Нет\\
Я очень небогат, не забывай\
Единоразовые пожертвования разве что\\

\textbf{Я}\\
Значит релоад мотивирует тебя больше работать

\textbf{В}\\
Нет, я ещё и ленивый

\textbf{Я}\\
Хотел бы ты видеть картинки/графики в игре?\
Например, диаграмма доходов/расходов твоего государства\
Или иллюстрации к вердам\\

\textbf{В}\\
Про картинки кстати можешь к Ангине обратиться она превосходно рисует

\textbf{Я}\\
А оно надо?

То есть да, выглядеть будет стильно

\textbf{В}\\
Не знаю, вот я тебя и спрашиваю

Там от ресурсов зависит и всего такого\
По моему мнению\\
Доходы + 100\\
Расходы - 50\\
Достаточно

И там зависит от ресурсов системы и как там считается

\textbf{Я}\\
Понял

Как думаешь, что именно будет доставлять тебе наибольшее количество эмоций в релоаде? Общение с игроками/верды/написание приказов/что-то еще?

\textbf{В}\\
Верды и сам игровой процесс

\textbf{Я}\\
Отлично

Ну тогда все по вопросам

Пиши, если будут идеи!