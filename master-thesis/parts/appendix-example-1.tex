Ссылка на приложение~\ref{appendix:2}.
\textbf{Опрос потенциального пользователя А.}
\begin{tabbing}
\hspace{3cm} \= \hspace{10cm} \= \kill
\textbf{Я:} \> Появился неиллюзорный шанс начала релоада в следующем году. \\
\textbf{А:} \> Тебя уволили? \\
\textbf{Я:} \> Нет, но я осознал, что мне нравится делегировать. И есть человек, \\
      \> который готов написать нормального бота для релоада. \\
\textbf{А:} \> Так релоаду мешало отсутствие бота. \\
\textbf{Я:} \> Возможно, я привлеку к этому проекту кого-нибудь еще. \\
\textbf{А:} \> Я думал, тебе лениво разбираться, нет? \\
\textbf{Я:} \> Конечно лениво. Поэтому я буду активно использовать GPT. \\
\textbf{А:} \> Потенциальный пользователь выразил удивление. \\
\textbf{Я:} \> Продолжаю серию вопросов про релоад. \\
\textbf{А:} \> Вопрошай. \\
\textbf{Я:} \> Тебе было бы удобнее играть в ВК или в ТГ? \\
\textbf{А:} \> В ТГ. \\
\textbf{Я:} \> \textit{(удивление)} \\
\> Почему так? \\
\textbf{А:} \> Возможно, релоад уже не стоит проводить, \\
\> ожидания перегреты. \\
\textbf{А:} \> Разметка. \\
\textbf{Я:}\\
Да, в перспективе на каждого персонажа по ассистенту\\
Что такое оаи агент?\\

\textbf{А:}\\
Пусть они друг в друга через некий буфер кидают инфу\\
Ну, они вроде хотели сделать систему персонализированных ассистентов\\

\textbf{Я:}\\
Ваще не гуглится\\

\textbf{А:}\\
Хмммм\\
Странно, я помню новости\\
Ну, мб это было в другом мире\\

\textbf{Я:}\\
Ааа оаи это openai\\
Тогда да\\
\textbf{Я:}\\
Кароч буду валидировать с тобой свои мысли по поводу релоада\\
Самой большой сложностью в этой задаче я вижу как раз размер контекстного окна\\
Типа если Никите в 7 классе было норм держать в голове ситуацию в 15 странах одновременно, то для гпт ето пока анрил\\
Поэтому думаю сделать оркестр ассистентов, которые будут переговариваться друг с другом\\
Типа один ассистент на всю игру (около ста лет), в которого будут закидываться только краткие сводки за последние 10 лет\\
Один отвечающий за десятилетие\\
Один за год (то есть за игровой день)\\
Еще есть сложность во взаимодействии ассистентов. Нужно как-то мотивировать игроков переговариваться друг с другом, устраивать всякие саммиты и отправлять послов. Но это мб на первом этапе можно и руками делать, все равно я в ручном режиме буду все контролировать\\

\textbf{А:}\\
Насколько я помню ты буквально хотел, чтобы советники и прочие поцы были отдельными личностями\\
Звучит как очевидная, максимально учебниковая задача для ОАИ Агенстс\\
\textbf{А:}\\
Нет\\
Донат рушит геймплей, по моему мнению\\

\textbf{Я:}\\
А если они не будут влиять на твое преимущество? Типа не чипы инфокоммунизма, а скин? (ред.)\\

\textbf{А:}\\
Э... Что?\\

\textbf{Я:}\\
Ну типа что-нибудь мелкое, но прикольное, созданное скорее с целью поддержать разработчиков. Как цвет оружия в кс\\

\textbf{А:}\\
Йоу, это же текстовая рпг\\
Я могу буквально делать что хочу\\
В этом и прелесть\\

\textbf{Я:}\\
Ну типа что-нибудь мелкое, но прикольное, созданное ск...\\
\textbf{А:}\\
Ну...\\
Сложно сказать\\
\textbf{Я:}\\
\textit{Андрей Шмушев}\\
Таки релоад и не будет в том виде, в каком я его планировал\\
То, как я сейчас вижу релоад — это скорее MVP\\
Вопрос третий, был бы ты готов платить за релоад? Сколько рублей в неделю?\\

\textbf{А:}\\
Если он будет дарить мне те самые эмоции от игры во Внеземную — полагаю, около 2000\\

\textbf{Я:}\\
В неделю? Раньше ты вроде говорил про 2000 в месяц\\

\textbf{Я:}\\
В неделю? Раньше ты вроде говорил про 2000 в месяц\\
\textbf{А:}\\
Да, в неделю. Инфляция :)\\

\textbf{Я:}\\
Чудеса\\
\textbf{Я:}\\
Ты бы предпочел, чтобы я нанял Водяна и вердикт выносил он?\\

\textbf{А:}\\
\#блин\\
Я не знаю\\

\textbf{Я:}\\
Ха-ха\\
Ладно, я тебя понял\\
Не знаю, мне хотелось бы, чтобы в релоад могли поиграть много людей, хотя бы 20 человек\\

\textbf{А:}\\
\textit{Я}\\
Ты бы предпочел, чтобы я нанял Водяна и вердикт выносил он?\\
Я не позволю Водяну прикасаться к Инферии как судье (ред.)\\
Так что лучше GPT\\

\textbf{Я:}\\
Эээе, безальтернативный выбор сработал\\
Хотя почему безальтернативный\\
У тебя всё ещё есть возможность нанять меня как судью за 50к в месяц!\\

\textbf{Я:}\\
Очевидно, я еще не придумал как монетизировать, но можно же как-нибудь\\

\textbf{А:}\\
Сейчас ответ нет, но возможно ты придумаешь что-то, с чего я удивлюсь и скажу да\\
Не знаю, возможность нанять кого-то, кто будет рассказывать тебе полный лор Инферии, переодевшись в персонажа из Викии Третью\\

\textbf{Я:}\\
Ха-ха)))\\
Ну, типа\\
Окей, следующий вопрос\\
Насколько, по твоим оценкам, игра, использующая GPT, принесет тебе меньше эмоций?\\
Допустим, не полный GPT, а я буду использовать его для генерации текста или картинок\\

\textbf{А:}\\
Это не проблема само по себе, особенно если будет судья на человеке. Я опасаюсь GPT из-за того, что она тяготеет к среднему, а Инферия — это буквально государство-экстремум.\\
И она может типа\\
НЕ ПОНЯТЬ ИДЕЮ\\
УПУСТИТЬ СУТЬ\\

\textbf{Я:}\\
Суть\\
Насколько тебе нравится рандом в ВПИшках? Хотел бы ты, чтобы он был в релоаде? Как сильно влиял бы?\\

\textbf{А:}\\
Не нравится. Не хотел бы. Влиял на косметику, типа "Небо сегодня было пасмурным, Викия грустно смотрела в окно". Псевдослучайные штуки, типа катаклизмов — ок.\\

\textbf{Я:}\\
\textit{абстрактные рассуждения}\\
Ну вот, не знаю, насколько ты тут прав. Когда мы делаем что-то, не существовавшее раньше, очень сложно рассчитать вероятность успеха. Условно, когда кто-то вкладывает деньги в проект, он вряд ли может точно оценить, получится ли он вообще.\\

Поэтому я думаю сделать что-нибудь типа рандома с нормальным распределением, на параметры которого можно будет влиять в процессе игры. Типа вкладываешься в навык, получаешь сдвиг колокола вправо. Значит, вероятность прорывных открытий будет больше (но всё ещё не 100\%).\\
\textbf{Я:}\\
Короче, я как-то пытаюсь добавить субъективность в игру, потому что GPT ей не обладает. Пока все мои эксперименты — это типа:\\
– разработать телепорт\\
– спустя пять лет упорного труда ваши ученые разработали телепорт!\\

\textbf{А:}\\
Да, однако это всё-таки факторная штука.\\
То есть, не знаю, Зимбабве не разработает нечто подобное\\
Вообще\\
Какая бы удача не случилась\\

\textbf{А:}\\
Да, звучит круто\\

\textbf{А:}\\
Кстати, это может интересно работать с разными отраслями (ред.)\\

В смысле, колокол же может не только двигаться вправо и влево\\
Но и становиться более или менее дисперсным\\

В смысле, полагаю, обильное финансирование "науки в целом" — это про утолщение правого хвоста.\\
То есть, про некую систему, улавливающую таланты и раскрывающую их в разных сферах\\

А вот финансирование системы образования, полагаю, будет именно что двигать колокол, потому что талантов будет становиться больше в среднем\\

\textbf{Я:} Как ты относишься к генерации картинок в ВПИ?\\

\textbf{А:}\\
Идея сомнительная.\\
В смысле, тебе надо или прям мастерски владеть генерацией картинок,\\
чтобы на сотне изображений персонаж выглядел одинаково, и ситуация учитывалась\\
Либо лучше не надо.\\

\textbf{Я:}\\
А кто сказал, что везде будет один и тот же персонаж?\\
Типа там текст про строительство завода и там рабочие\\
Или митинг какой-нибудь\\

\textbf{А:}\\
Ну, типа\\
Здания Инферии, например, вероятно, будут сильно отличаться от зданий другого места\\
Причем они будут в чём-то похожи друг на друга\\
При этом всё ещё функциональны\\
А вообще\\
Можно просто сделать эту фичу отключаемой\\

\textbf{Я} \\
Ну вот жиз, для тебя думаю сделать максимально настраиваемую игру \\
Чтоб ты например мог вообще отключить ГПТ при желании \\
Кста есть идея добавить расшифровку голосовых сообщений. Чтоб можно было не писать ручками долгие роллы \\
Ты бы пользовался?

\textbf{А} \\
Ты же знаешь, что у яндекс.клавиатуры отличны голосовой ввод? \\
Но вообще да, я бы пользовался. \\
Ты же знаешь, что у яндекс.клавиатуры отличны голосовой ввод? \\
\textbf{Я} \\
Нет
\textbf{Я} \\
Шамш, было бы тебе интересно в релоаде смотреть на графики? \\
Типа курс джибриллита к митолу или диаграмма доходов/расходов Инферии

\textbf{А} \\
ДА (ред.) \\
ДА (ред.) \\
Мой движок для впи который я все думал создать \\
Должен был базироваться как раз на графиках
\textbf{Я} \\
Скажи, какие ты показатели хочешь видеть на графиках? Пока думаю ВВП, казна, численность населения, может что-то ещё?

\textbf{А} \\
Зависит от того, насколько ты готов в это углубляться. \\
К тому же, не забывай, что Инферия — это плановое государство, у него могут быть разные странные графики. \\
Но если бы мне в реальном времени показывали АК-модель... \\
\url{https://ru.m.wikipedia.org/wiki/АК-модель}

Проблема в том, что она рыночная. \\
А мы не экономисты, чтобы адаптировать сложную модель так сразу. \\
Так что может быть достаточно и графика остатка Солоу, \\
такого как "Общая факторная производительность", или "Мера нашего незнания", также известная как "Коэффициент прогресса". \\
Грубо говоря, это чтобы я мог отслеживать степень инновационности экономики. \\
Поскольку, как известно, остаток Солоу выражает инновационность экономики. И только в СССР из всех сверхдержав он был отрицательным.\\
\textbf{Я} \\
Как ты относишься к рамкам эпохи? \\
Я помню, что плохо, но обоснуй

\textbf{А} \\
Ну \\
Прогресс — это эндогенное явление общества. \\
Проблема долгих эпох прошлого — в их эволюционности развития. \\
Типа, сделать что-то в феодализме очень сложно. \\
Микроулучшения накапливаются столетиями, и этим улучшениям активно противодействуют. \\
Фактически, существует ряд условий, в которых эта ситуация переламывается. \\
И если эти условия возникают, то она изменяется. \\
А если не возникают, то она может сохраняться тысячелетиями. \\
Из-за чего идея "Сейчас эпоха X, вы не можете перейти в эпоху Y в ближайшие n лет, потому что никак" плоха. \\
Мы играем в ВПИ, чтобы желать: "А вот если бы у Петра Первого был учебник по макроэкономике..."
\textbf{Я} \\
Пхпхпхх \\
Ну вот кажется, я такое не хочу. \\
То есть если ты в начале игры скажешь "разработать ядерку", тебе ответят "что такое ядерка".

\textbf{А} \\
Впрочем, Золотой Полувек — это уже не та эпоха, где ты хочешь быстрее пройти этот период с мечами. \\
Типа, это же период средней индустриализации. Самое интересное.

\textbf{Я} \\
Короче, пока я думаю так: ограничения эпохи будут, но я их буду активно менять. Допустим, если твои ученые изобретут электричество (а тут, как ты помнишь, влияет рандом), то это уже начало следующей эпохи.

\textbf{А} \\
Стоп, погоди. \\
Где мы стартуем?

\textbf{Я} \\
Ну либо GPT ошибётся, и в самом начале игры пришлет ивент типа "ваши ученые изобрели ядерку". \\
Конец Золотого Полувека.

\textbf{А} \\
В смысле, ээ... Если помнишь, Алинэхо получила своё крыло именно за разработки в электричестве.
\textbf{Я} \\
(Изображение с пометкой "БЛИН")

\textbf{А} \\
То есть, это уже после того, как инферцы запустили кек-ракету на механике.

\textbf{Я} \\
Интересно, успел ли GPT обучиться на твоих книгах.

\textbf{А} \\
Причем, на десятилетия после.

\textbf{Я} \\
То есть ты бы хотел, чтобы я следовал лору ДПвС?

\textbf{А} \\
Ну, даже если отбросить ДПвС и взять чистый лор Инферии, то кек-ракета уже была.
\textbf{Опрос потенциального пользователя Т.}
\textbf{Т} \\
Рассудительная боевка — круто. \\
Чем больше интерактива, тем веселее.

\textbf{Я} \\
Что такое рассудительная? \\
Типа как в шахматах?

\textbf{Т} \\
То, что ты Алёне писал. \\
С попыткой описать тактику и стратегию в ключевые моменты. \\
Война — дело долгое, а потому наполненное событиями.

\textbf{Я} \\
Хммм \\
Ну вот у меня пока ноль идей, как это с GPT-ассистентами реализовать. Но буду думать. \\
Короче, ты бы выбрал боевку?

\textbf{Т} \\
Да \\
Как много вердов ты бы хотел? ВЕРД — "вердикт" от админа. Например, ты пишешь "построить золотой мост". Админ отвечает: "у вас нет золота". Это верд.

\textbf{Т} \\
А второй можно, если ты отмел этот?

\textbf{Я} \\
Несколько раз в день/раз в день/раз в несколько дней/чаще?

\textbf{Я} \\
\textbf{Т} \\
А второй можно, если ты отмел этот? \\
Как понять отмел?

\textbf{Т} \\
\textbf{Я} \\
Как понять отмел? \textit{Сообщение}
\textbf{Т} \\
\textbf{Я} \\
Как понять отмел? \textit{Сообщение} \\
Ну вот ты сказал, что золота нет — я сижу и грущу до следующего или могу передумать? \\
Или это последствия?

\textbf{Я} \\
Аа, понял тебя. \\
Ну вот это еще один вопрос к разработке, надо подумать.

\textbf{Т} \\
«Как много вердов ты бы хотел? ВЕРД — "вердикт" от админа. Например, ты пишешь "построить золотой мост". Админ отвечает: "его спиздила Алёна". Это верд» \\
Так правильнее. 🔥🦔

\textbf{Я} \\
В идеале я бы сделал, что игрок пишет сколько угодно приказов, и раз в сутки GPT ему отвечает на все. Но мб это будет слишком дорого. \\
В связи с этим следующий вопрос — сколько бы ты был готов платить за участие в игре?

\textbf{Т} \\
\textbf{Я} \\
В идеале я бы сделал, что игрок пишет сколько угодно... \\
Ну так слишком много будет действий?
\textbf{Я} \\
\textbf{Т} \\
Ну так слишком много будет действий? \textit{Сообщение} \\
В чем минусы? Это же не "игра по шагам", здесь нет ходов. В этом и плюс, что в ВПП ты можешь писать сколько хочешь, и про что хочешь. Полная свобода.

\textbf{Т} \\
\textbf{Я} \\
В связи с этим следующий вопрос — сколько бы ты бы... \\
300 в неделю норм, даже 500 норм \\
После 1000 перебор, наверное. 👍🎅

\textbf{Я} \\
\textbf{Т} \\
Ну так слишком много будет действий? \\
Тогда вопрос, это хорошо или плохо?

\textbf{Т} \\
\textbf{Я} \\
В чем минусы? Это же не "игра по шагам", здесь нет ... \\
Надо будет тогда самому себя контролировать, чтобы не развить летающих робо-котов за день.

\textbf{Я} \\
Не получится :) (надеюсь, у меня удастся ограничить GPT, чтобы он такого не допускал).
\textbf{Т} \\
И все их за день написать в GPT

\textbf{Т} \\
\textbf{Я} \\
Не получится :) (надеюсь, у меня удастся ограничить GPT) \\
Так норм

\textbf{Я} \\
Ну то есть все равно будут рамки эпохи, если мы живем в стимпанке, то за год ты электронный компьютер не соберешь. 👍👨‍💻

Пока я планирую сделать это так: на каждое действие под капотом GPT будет оценивать, как много времени это займет. И ставить таймер \\
1 день = 1 год

Допустим, ты пишешь приказ: начать разработку квантовых компьютеров. GPT думает, ага, чтобы в стимпанке создать квантовый комп, нужно 150 лет. Тогда результаты твоего исследования придут только через 150 дней.

Может найду более простой вариант, и сделаю это как-то по-другому!

\textbf{Т} \\
Звучит сложно по промптам \\
Но готов помочь
\textbf{Я} \\
Таки да, тут основная сложность проекта в промптах

\textbf{Т} \\
У меня есть опыт

\textbf{Я} \\
\textbf{Т} \\
Но готов помочь \\
Фронт и промпт похожие слова, поэтому ты изучил все?

\textbf{Я} \\
\textbf{Т} \\
У меня есть опыт \\
Ого кайф

Расскажи, есть в элизе возможность создавать ассистентов со своими промптами? Я не нашел

\textbf{Т} \\
\textbf{Я} \\
Фронт и промпт похожие слова, поэтому ты изучил в... \\
Ага. Перепутал. 😂🎅

\textbf{Т} \\
\textbf{Я} \\
Расскажи, есть в элизе возможность создавать ассистентов...\\

Вроде нет, в генезисе вроде да\\
Как ты относишься к рандому в играх? Он тебя скорее бесит из-за дисбаланса или нравится?\\

\textbf{Т} \\
Ну от количества зависит \\
Он конечно нужен, но в соотношении 7:3 где 7 это действия игроков \\
(Очень примерно)\\
\textbf{Я} \\
Поняв \\
А вот если игрок не играет \\
Допустим, завал на работе, и нет времени на игру \\
Че делать?

\textbf{Т} \\
Ну смотря сколько времени он будет занят?

Если один день не отвечает, то и ладно, застой в экономике, бывает\\

Если неделю, то грустно\\
\textbf{Я} \\
Услышал тебя \\
Ну кстати здесь твое мнение прям сильно расходится с большинством пххп

\textbf{Т} \\
А у остальных какое?

\textbf{Я} \\
Собственно основной бомбеж Гоги на меня за Внеземную был в том, что у него полстраны захватили, пока он был АФК

\textbf{Т} \\
Хах

\textbf{Я} \\
И он бы хотел, чтоб за него играл ИИ в это время

\textbf{Т} \\
Ну я и говорю, что один день можно простой сделать \\
Потом надо или ждать или как-то еще решать

\textbf{Я} \\
Мб мб

\textbf{Т} \\
\textbf{Я} \\
И он бы хотел, чтоб за него играл ИИ в это время \\
Это сложно, ибо он может накосячить сильно

Вообще, я для себя не вижу проблемы выделить 15 минут прочитать что произошло и написать пару действий)\\
Но это мое отношение

Что если подписался играть, то уж будь добр)

\textbf{Я} \\
Ну хз насчет 15 минут

Если будет много срачей в беседке, то кто-то может захотеть их тоже читать

Но да, на 1 уровне как будто 15 минут в день норм

\textbf{Т} \\
Ну я ж не играл

\textbf{Я} \\
Как тебе было бы интересней общаться с другими игроками? Внутри бота или в беседке?

Я имею в виду можно в боте сделать функцию "отправить телеграмму другому правителю" \\
И писать через нее

А можно самому пойти в личку/в беседку и обсудить с ним вопрос

\textbf{Т} \\
\textbf{Я} \\
А можно самому пойти в личку/в беседку и обсудить... \\
скорее так прикольнее

я за общение)\\
\textbf{Я} \\
Было бы тебе удобно использовать голосовые сообщения для написания приказов?

\textbf{Т} \\
скорее нет, но в качестве поддержки Алёны, вещь нужная)

\textbf{Я} \\
Пхпхпхх

\textbf{Т} \\
я просто думаю, что не успею подумать за время голосового \\
поэтому проще писать

\textbf{Я} \\
Жиз жиз

Хотел бы ты видеть картинки/графики в игре?

\textbf{Т} \\
да, думаю будет мемно)

\textbf{Я} \\
Допустим, диаграмма расходов/доходов твоего государства

\textbf{Т} \\
хохох \\
я больше подумал про иллюстрации к действиям \\
я скорее всего на диаграммы забью, ибо я тупенький и мне они ничего не дадут)

\textbf{Я} \\
Поняв
Иллюстрации к вердам/новостям тоже тема

\textbf{Т} \\
ага

\textbf{Я} \\
Как ты думаешь, что в большей степени будет влиять на твои эмоции во время игры?

Например, процесс обдумывания/написание приказов/чтение вердов/общение с другими игроками/просмотр картинок/что-то еще?

\textbf{Т} \\
ох ты ж \\
я очень надеюсь, что получится соблюсти баланс между мемичностью и тем, чтобы игровой процесс имел смысл)

мне, конечно, будет интереснее всего продумывать множество вариантов гениальных планов по развитию и захвате власти и влияния

но в целом все вышеперечисленное мне в радость 😊❤️

\textbf{Я} \\
Понял, годнота \\
А как ты относишься к исходному дисбалансу? Типа у всех разные территории/доходы/население

\textbf{Т} \\
ну в пределах разумного, но в рамках сеттинга так и должно быть же)
\textbf{Опрос потенциального пользователя Ж.}\\
Тебе было бы удобнее играть вк или в тг?

\textbf{Ж} \\
поставил ❤️🎅

\textbf{Я} \\
Тебе было бы удобнее играть вк или в тг?

\textbf{Ж} \\
тг

\textbf{Я} \\
Как ты относишься к использованию ГПТ в ВПИ? Как думаешь, это бы ухудшило или улучшило твои эмоции от игры?

\textbf{Ж} \\
\textbf{Я} \\
Как ты относишься к использованию ГПТ в ВПИ? Как ... \\
сложно сказать, может случиться так что и мастер и игрок - боты

\textbf{Я} \\
АВЩПХВПХЭЩВП да, будет шедевр

\textbf{Ж} \\
В этом есть + и - \\
нельзя сказать, что это точно плохая идея, но нельзя сказать и обратного

\textbf{Я} \\
Какие ты видишь минусы?
\textbf{Ж} \\
- могут быть в том, что бот может в какой-то момент вести 2 разные ветки событий для разных игроков \\
но я не уверен в этом \\
Смотря насколько глубоко будет внедрен бот

ну и я не знаю, как ты все настроил \\
Мб будет мега круто

\textbf{Я} \\
Ну жиз, тоже об этом думаю. Скорее всего, будет какой-нибудь оркестр ассистентов. Типа один для инферии, другой для ХФ, третий для взаимодействия между ними

Пока никак не настроил, вот кастдев провожу!

\textbf{Ж} \\
Это твой диплом?)

\textbf{Я} \\
Ахзщващпх не

\textbf{Ж} \\
или курсач? \\
понял(

\textbf{Я} \\
Исключительно (((бизнес-проект)))

Сколько бы ты был бы готов платить в неделю?

\textbf{Ж} \\
понятно
у инферии опять будут чипы в каменном веке

\textbf{Я} \\
Скорее всего, будет три варианта игры \\
Бесплатная, с низким уровнем погружения \\
Средняя, с более детальным \\
Высокая, для Шамша

Разница в количестве вердов в день

\textbf{Ж} \\
\textbf{Я} \\
Сколько бы ты был бы готов платить в неделю? \\
хм \\
я бы попробовал бесплатный, потом например на дня средний, если понравится, то остался бы на среднем

Нужно явно показать разницу между средним и бесплатным

сколько готов платить? \\
ну цена за бота меня устраивала)) 😂🎅

\textbf{Я} \\
Пока не продумывал точно, но, например 1 верд в день или 5

\textbf{Ж} \\
но то в месяц было

\textbf{Я} \\
\textbf{Ж} \\
сколько готов платить? ну цена за бота меня устра...\\
\textbf{Ж} \\
сколько готов платить? ну цена за бота меня устра...

То есть 500 рублей в месяц?

\textbf{Ж} \\
ну типа того

пока что хз

\textbf{Я} \\
Оки! Как ты относишься к рандому в ВПИ?

\textbf{Ж} \\
думаю, что он нужен, но его должно быть не очень много

Temperature = 0.5

\textbf{Я} \\
Хехехех

А какие ты бы хотел боевки в ВПИ?

Хотел бы участвовать в войнах сам или приказывать генералам?

\textbf{Ж} \\
генералы - ии или игроки?

\textbf{Я} \\
Пока хз, будет ли именно проработка по личностям, но скорее ИИ

\textbf{Ж} \\
хм

ну я думаю. что и так и так было бы классно
ну я думаю, что и так и так было бы классно

ты можешь отправить генерала в бой

а можешь и сам возглавить войско 👍🎅

\textbf{Я} \\
Что бы мотивировало тебя общаться с другими игроками? Типа не просто отправить послов, а написать в беседку/личку

\textbf{Ж} \\
думаю, что обсуждение условий союзов, контрактов и войны

но это лучше сделать через какой-то чат с ботом, который будет все фиксировать

\textbf{Я} \\
Почему так? В личку неудобно?

\textbf{Ж} \\
Это можно будет добавить в общий контекст

И в случае чего с этим связать например утечку

Если к тебе кто-то шпионов отправил или подобное

Это как один из вариантов, почему так лучше

\textbf{Я} \\
Нормас нормас
\textbf{Никита} \\
Привет! Планирую в следующем году запускать релод. \\
Игра будет идти примерно полгода. Будешь участвовать? \\
Хочу задать несколько вопросов по ожиданиям 🔥🎅

\textbf{Георгий} \\
Привет, а если буду тормозить с ответами, это тебе может испортить админство? Или "если вы афкшили пару дней и вам за это время накидали ядерок - ваши проблемы"?

\textbf{Никита} \\
Хммххм, ето проблема, которую мне нужно заранее учесть. Скорее всего, события в твоей стране будут происходить и без твоего участия, ибо это большая система с кучей людей. Тебе бы такое не нравилось?

\textbf{Георгий} \\
Я думаю, что из двух зол это меньшее (т.е. я бы предпочел так). Поиграть и пофантиться - с удовольствием, хочется. Но не могу гарантировать, что не будет условно какого-то завала по работе, что мне не очень захочется ещё думать "бля, а как бы сходить получше", и не хочется чтобы приходилось меня тегать и ждать, когда я там пойду. Поэтому, если такая ситуация будет, я пойму и приму её

\textbf{Никита} \\
В прошлом моем подходе к созданию релода, я собирался ввести три режима игры, различающимися уровню проработки. Условно, верды раз в день, раз в час и раз в минуту.
Кажется, если бы такое реализовать, твоя проблема была бы решена? (типа пусть там Шамш все обсасывает и детально прорабатывает, а ты прост когда удобно большими мазками направляешь Лурк)

\textbf{Георгий} \\
Да, звучит достаточно разумно. Я бы, правда, сказал 2-3 раза в день, но это, наверное, адаптивная фигня

\textbf{Никита} \\
Оки. И все-таки, как бы ты отреагировал, если в твое отсутствие тебя разбомбили ядеркой? 😁🎅

\textbf{Георгий} \\
Я думаю, что некоего "идеального" варианта тут не получить, но в целом, если у моей страны нет никакого ПРО и всё такое, ну как-то кринжово если я пишу "строим ПРО и защищаемся", оно же не должно вот так вырастать из ниоткуда, когда нужно. Т.е. если я заранее не позаботился об этом - это моя проблема, и я бы не был недоволен

При этом, конечно, если оно таки есть, оно должно работать без моей команды, мне надо писать "используйте ПРО", понятно, что его надо использовать

Ну и так себе, если у нас вечером было средневековье, я весь день не отвечал, а всякие инферцы за день со средневековья прыгнули на ядерки и все разбомбили, но это тоже воспринимается как какая-то аномальная нереальная ситуация

Так что, скорее нормально восприму. Я достаточно доверяю многуважаемому админу, чтобы принимать, что если такое произошло - тому только моя вина 😁🙏
\textbf{Никита} \\
Поняв! Тебе было бы удобнее играть в тг или вк?

\textbf{Георгий} \\
+- одинаково, спокойно и там, и там буду

Но чуть удобнее, наверное, было бы в тг, я там вероятнее уведомление увижу

\textbf{Никита} \\
Окей. Как ты относишься к использованию GPT в ВПИ? \\
Как думаешь, насколько сильно ето скажется на качестве вердов?

\textbf{Георгий} \\
Скорее, положительно, но: \\
* лучше хотя бы прочитать, что он нагенерил разок) (в частности, думаю, важна проверка контекста - ллмка должна помнить, что у меня есть ПРО! Вот это, наверное, самый неприятный вид экспириенса, когда судья не учёл какую-то такую фигню, и как бы оспаривать тяжело, потому что все играют из той позиции, что тебя уже разнесли ядеркой) \\
* не хочется, чтобы совсем уж без великого админского креатива) Но даже если условное 100\% ГПТ, было бы любопытно попробовать и так

\textbf{Никита} \\
Кайф. Собсна тут возникает проблема, что ГПТ4о платный. И как оказалось, дорогой, Женя за месяц использования моего бота потратил ~1500 рублей.

В связи с этим вопрос. Был бы ты готов платить за игру? Сколько рублей в неделю?
\textbf{Георгий} \\
Думаю, 300-500 рублей в месяц

\textbf{Никита} \\
Ну жиз, собсна думаю сделать три режима игры. Бесплатный, средний и дорогой. И в зависимости от режима разное ограничение на число вердов в день

Кажется, вовлеченность человека в игру коррелирует с числом денег, которые он готов платить

Насколько тебе нравится рандом в ВПИ? Хотел бы ты, чтобы он был? Или все как во внеземной, все по решению админа?

\textbf{Георгий} \\
Сложно сказать. Но, думаю, будет хорошо, если хоть какое-то кол-во рандома будет, так будет потенциально меньше априори недовольств решениями админа

\textbf{Никита} \\
Жиз жиз

Есть идеи, как реализовать боевку кста?

Прост для меня боевка во внеземной и так была идеальной. Путин же не ездит на фронт, он только отдает приказы генералам

Но видимо игрокам хочется какого-то экшена хз
Ну кароч ты бы предпочел, чтобы бои за тебя вел ИИ

\textbf{Георгий} \\
Ну, кстати, если на рофельном уровне, то можно даже

\textbf{Георгий} \\
\textit{Ну кароч ты бы предпочел, чтобы бои за тебя вел ИИ}

Именно "сражаться", думаю, да

Было бы прикольно, на мой взгляд, участие в боевке, ну, например, на таком уровне, что я знаю, что у меня дофига лесистая страна, и я там даю тактико-стратегическое распоряжение загонять врагов вглубь леса, фигачить ловушки, гуки на деревьях, все дела

Или, например, если я придумаю рыть вокруг замков ров и заполнять его голодными свиньями

\textbf{Никита} \\
Суккк))

\textbf{Георгий} \\
Т.е. это такой же верд, только относящийся к "военной" сфере, а не прямо управление боевыми действиями

\textbf{Никита} \\
Годно годно, хотя хз пока, как это реализовать с GPT

В общем спасибки, если будут идеи, пиши! 👌
\textbf{Никита Пеганов} \\
Тебе было бы удобнее играть вк или в тг?

\textbf{Даниил Воронов} \\
Без разницы

\textbf{Никита Пеганов} \\
А как много вердов ожидаешь?

Типа несколько раз в день, раз в день, раз в неделю хз

\textbf{Даниил Воронов} \\
Поскольку ты гений я рад даже маленькому вердику

\textbf{Никита Пеганов} \\
Верд: ты гигачад
\textbf{Никита Пеганов} \\
Как ты относишься к GPT для написания вердов?

\textbf{Даниил Воронов} \\
Хз сомневаюсь что ты сможешь её научить что-то чтобы она выдавала чето на твоём уровне

\textbf{Никита Пеганов} \\
Посмотрим жиз

короче ваша армия разбита

\textbf{Даниил Воронов} \\
Типо она не сомневаюсь сможет сделать красивый текст в ответе или типо того

\textbf{Никита Пеганов} \\
Как ты относишься к рандому в ВПИ/ролках?

\textbf{Даниил Воронов} \\
Но это хз будет как все продолжения сомнабул

\textbf{Никита Пеганов} \\
А че там в продолжении сомнабулы?

\textbf{Даниил Воронов} \\
Положительно

Но это хз будет как все продолжения сомнабулы
\textbf{Никита Пеганов} \\
Как бы тебе было бы удобнее общаться с другими игроками? В беседке или через бота?
(через бота - можно сделать в боте кнопку "написать письмо другому правителю" и отправлять напрямую другому игроку)

\textbf{Даниил Воронов} \\
Через бота
\textbf{Даниил Воронов} \\
И желательно абсолютная анонимность игроков

Потому что иначе метаейство будет

\textbf{Никита Пеганов} \\
Чтобы не приехали в казахстан пиздиться

\textbf{Даниил Воронов} \\
Ахахахахахах

\textbf{Никита Пеганов} \\
\textit{Потому что иначе метаейство будет}

В чем минусы? Сообщение

\textbf{Даниил Воронов} \\
Скучно и не рпшно

\textbf{Никита Пеганов} \\
Ну например ТСГЕМ во внеземной был организован в личке

Типа я сам постфактум узнал, что он появился

По-моему было прикольно

\textbf{Даниил Воронов} \\
Ну он именно в личке
\textbf{Никита Пеганов} \\
Но оки, значит, сделаем ботика с анонимностью!

\textbf{Даниил Воронов} \\
Это типо личные переговоры вся хуйня

\textbf{Никита Пеганов} \\
Жиз жиз

\textbf{Даниил Воронов} \\
А флуд чат именно что начинает метаейство и погружение портит

\textbf{Никита Пеганов} \\
Как бы ты хотел участвовать в войнах? Прост отдавать приказы генералам или сам расставлять силы, вести армию вперед?

\textbf{Даниил Воронов} \\
От системы зависит, мне кажется второй вариант будет тебе гемморнее считать

Ну либо это будет странная версия камень ножницы бумага

Делай тут как тебе самому удобно водить

\textbf{Никита Пеганов} \\
Да пофиг, сделаем

Был бы ты готов платить налоги за участие в релоаде? Сколько рублей в неделю?
\textbf{Даниил Воронов} \\
Могу нбсды кидать

А так хз

\textbf{Никита Пеганов} \\
Уву

Принято

\textbf{Даниил Воронов} \\
Я нищий оч не забывай

Единоразовые донаты разве что

\textbf{Никита Пеганов} \\
Значит релоад мотивирует тебя РАБотать больше

\textbf{Даниил Воронов} \\
Не-а я ещё и ленивый

\textbf{Никита Пеганов} \\
Хотел бы ты видеть картинки/графики в игре?

Например, диаграмма доходов/расходов твоего государства

Или иллюстрации к вердам
\textbf{Даниил Воронов} \\
Про картинки кста можешь к Ангине обратиться она ахуенно рисует

\textbf{Никита Пеганов} \\
А оно надо?

Ну то есть да выглядеть будет стильно

\textbf{Даниил Воронов} \\
Хз вот я тебя и спрашиваю

Там от ресов зависит и всего такого

Имхо \\
Доходы + 100 \\
Расходы - 50 \\
Достаточно

Ну и там зависит от ресов системы и как там считается

\textbf{Никита Пеганов} \\
Понял

Как думаешь, что именно будет доставлять тебе наибольшее количество эмоций в релоаде? Общение с игроками/верды/написание приказов/что-то еще?
\textbf{Даниил Воронов} \\
Верды и сам игровой процесс

\textbf{Никита Пеганов} \\
Кайфы

Ну тогда все по вопросам

Пиши, если будут идеи!

\textbf{Даниил Воронов} \\
Про картинки кста можешь к Ангине обратиться она ...

\textbf{Никита Пеганов} \\
Жиз жиз подумаю, спасибо!
\end{tabbing}