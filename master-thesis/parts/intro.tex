Современные достижения в области искусственного интеллекта и, в частности, больших языковых моделей (LLM) открывают новые перспективы для традиционных интерактивных развлечений~\cite{brown2020language, ouyang2022training}. Одной из областей, где применение искусственного интеллекта имеет значительный потенциал, являются текстовые стратегические игры, такие как военно-политические игры (ВПИ) — жанр, сочетающий элементы стратегии, ролевой игры и коллективного сторителлинга~\cite{catalogwpg2023}.

ВПИ представляют собой текстовые игры, в которых игроки управляют сложными структурами (государствами, военными силами, политическими организациями) путем написания приказов, а судья (или вердер) оценивает эти приказы и формирует вердикты — текстовые описания результатов действий~\cite{wpg-definition}. Этот процесс требует от судьи глубокого понимания игрового мира, механик, а также способности генерировать связные и логичные повествования, что делает эту роль одной из самых трудоемких в организации игры. Кроме того, традиционно судья ограничен в скорости обработки приказов, что создает естественный «потолок» для темпа игры и количества участников.

Большие языковые модели, такие как GPT-4, демонстрируют впечатляющие способности к пониманию контекста, следованию инструкциям и генерации связных текстов~\cite{openai2023gpt4}. Эти характеристики потенциально позволяют им выполнять роль судьи в ВПИ, автоматизируя процесс создания вердиктов и значительно ускоряя игровой процесс. Однако использование LLM в таком качестве сопряжено с рядом технических и методологических вызовов, включая проблему галлюцинаций, ограничения контекстного окна и сложности в поддержании долговременной согласованности~\cite{bubeck2023sparks, liu2023evaluating}.

В данной работе представлена разработка мультиагентной текстовой стратегической игры на основе оркестрируемых языковых моделей — системы, использующей несколько специализированных LLM-агентов для выполнения различных функций судьи в ВПИ. Исследование включает как теоретическое обоснование подхода, так и практическую реализацию в виде работающего прототипа, протестированного реальными игроками. Особое внимание уделяется механизмам обеспечения целостности игрового мира, преодоления ограничений LLM и создания интуитивно понятного пользовательского опыта.

Работа основывается на междисциплинарном подходе, объединяющем методы искусственного интеллекта, игрового дизайна и нарративных исследований~\cite{yuan2022wordcraft, park2023generative}. Представленная система не только демонстрирует практическое применение современных LLM в новой предметной области, но и открывает перспективы для создания более масштабных и динамичных текстовых игр, доступных широкой аудитории.

\subsection{Актуальность темы}

Текстовые стратегические игры жанра военно-политических игр (ВПИ) занимают особую нишу в игровой индустрии, предоставляя уникальный опыт коллективного стратегического взаимодействия. Ключевым ограничением данного жанра является высокая зависимость от человека-вердера (судьи), который обрабатывает игровые ситуации и формирует нарративную основу игры~\cite{dtf2021}. Это делает организацию подобных игр трудоемкой, снижает их доступность и ограничивает масштабы игрового сообщества~\cite{dtf2021}.

Современное развитие больших языковых моделей (LLM) создает предпосылки для решения данной проблемы~\cite{cloudru2025}. Недавние исследования показывают, что автоматизация создания интерактивного контента с помощью ИИ может существенно изменить подход к разработке игр с текстовой основой~\cite{uniteai2025}. Особую актуальность приобретает архитектурный подход на основе мультиагентных систем, где различные аспекты игрового взаимодействия обрабатываются специализированными ИИ-агентами~\cite{wikipediaMAS2025}. Подобная оркестрация позволяет преодолеть ограничения единичных моделей через разделение обязанностей и специализацию~\cite{wikipediaMAS2025}. В случае ВПИ это особенно важно, поскольку игра требует компетенций в различных областях: политике, экономике, военном деле, дипломатии~\cite{dtf2021}.

Разработка мультиагентной системы для ВПИ также имеет значение в контексте растущего интереса к цифровой гуманитаристике и инструментам совместного повествования~\cite{wikipediaDH2025}. Исследования показывают, что комбинация человеческого творчества и ИИ-ассистирования открывает новые горизонты для коллективного творчества и обмена идеями~\cite{wikipediaAIArt2025}.

С практической точки зрения, создание автоматизированной системы для проведения ВПИ может возродить интерес к жанру, сделать его доступным для более широкой аудитории и заложить основу для новых форм социального взаимодействия в цифровых средах~\cite{uniteai2025}. В научном плане проект представляет ценность как исследование применимости мультиагентных систем для поддержания сложных нарративных структур и последовательных игровых вселенных~\cite{habrMAS2018}.

Таким образом, разработка мультиагентной системы на основе оркестрируемых языковых моделей для автоматизации ВПИ представляет собой актуальную задачу, решение которой способно обогатить как теорию искусственного интеллекта, так и практику игрового дизайна.

\subsection{Цель и задачи исследования}

Основная цель исследования заключается в разработке и оценке мультиагентной системы на основе оркестрируемых языковых моделей для автоматизации роли вердера в текстовых стратегических играх жанра ВПИ, а также в определении эффективности и практической применимости такого подхода для создания интересного и последовательного игрового опыта.

Для достижения поставленной цели сформулированы следующие задачи:

\begin{enumerate}
    \item Провести анализ существующих подходов к организации военно-политических игр и выявить ключевые аспекты, требующие автоматизации.

    \item Провести customer development с опытными игроками ВПИ для выявления их ожиданий, болевых точек и требований к автоматизированной системе проведения игр.

    \item Разработать архитектуру мультиагентной системы, включающую специализированных ИИ-агентов для различных аспектов игрового процесса (обработка приказов, проверка на соответствие эпохе, оценка экономических показателей, формирование вердиктов).

    \item Реализовать первичный прототип системы (RELOAD WPG) на основе единой языковой модели и провести его тестирование с реальными игроками для выявления ограничений и потенциальных улучшений.

    \item На основе полученной обратной связи спроектировать и реализовать усовершенствованную версию системы с применением:
    \begin{itemize}
        \item Локальных языковых моделей для снижения стоимости эксплуатации
        \item RAG-системы для минимизации галлюцинаций и точного доступа к информации
        \item Оптимизированных механик взаимодействия для улучшения игрового опыта
        \item Специализированной системы симуляции боевых действий
    \end{itemize}

    \item Провести сравнительный анализ первичного прототипа и усовершенствованной системы по критериям:
    \begin{itemize}
        \item Качество генерируемых вердиктов
        \item Устойчивость к галлюцинациям
        \item Способность поддерживать долгосрочную согласованность игрового мира
        \item Удовлетворенность пользователей
        \item Вычислительная эффективность
    \end{itemize}

    \item Определить принципиальные ограничения и возможности применения оркестрируемых языковых моделей в контексте автоматизации текстовых стратегических игр.

    \item Сформулировать рекомендации для дальнейшего развития ИИ-ассистированных текстовых игр на основе полученных результатов.
\end{enumerate}

Решение данных задач позволит не только создать функциональную систему для проведения военно-политических игр с минимальным участием человека-вердера, но и внести вклад в понимание того, как мультиагентные системы на основе языковых моделей могут применяться для создания сложных интерактивных нарративных сред.

\subsection{Объект и предмет исследования}

\textbf{Объектом исследования} являются текстовые стратегические игры жанра военно-политических игр (ВПИ), в которых игроки управляют государствами или иными сложными структурами посредством текстовых приказов, обрабатываемых и интерпретируемых вердером (судьей).

\textbf{Предметом исследования} выступает процесс автоматизации роли вердера с помощью мультиагентной системы на основе оркестрируемых языковых моделей, включая:

\begin{itemize}
    \item Методы организации взаимодействия между игроками и ИИ-вердером в текстовом формате

    \item Архитектурные решения для создания мультиагентной системы, способной эффективно обрабатывать и интерпретировать игровые приказы

    \item Способы оркестрации различных языковых моделей для выполнения специализированных функций в контексте игрового процесса

    \item Методы преодоления ограничений языковых моделей (галлюцинации, поддержание долгосрочного контекста, согласованность генерируемого контента) для создания качественного игрового опыта

    \item Принципы проектирования пользовательского интерфейса и взаимодействия с ИИ-вердером, обеспечивающие максимальную доступность и понимание игрового процесса

    \item Критерии оценки эффективности и качества автоматизированной системы проведения ВПИ с точки зрения игрового опыта и технической реализации
\end{itemize}

Исследование фокусируется на изучении баланса между творческими аспектами генерации контента языковыми моделями и необходимостью сохранения игровой логики, исторической или жанровой достоверности и общей связности игрового мира. Особое внимание уделяется выявлению оптимальных подходов к декомпозиции задач вердера между различными агентами мультиагентной системы для достижения максимальной эффективности и качества игрового процесса.

\subsection{Методология исследования}

В основу методологии исследования положен комплексный подход, сочетающий методы программной инженерии, искусственного интеллекта и пользовательского дизайна. Исследование разделено на несколько взаимосвязанных этапов, каждый из которых имеет свою методологическую основу.

На подготовительном этапе применяются методы анализа предметной области и customer development для выявления ключевых требований и ограничений в контексте автоматизации ВПИ. Используется метод экспертных интервью с опытными игроками и организаторами ВПИ, а также анализ существующих проектов в данной области. Для структурирования полученной информации применяется методология Jobs-to-be-Done~\cite{christensen2016knowing}, позволяющая выявить основные потребности пользователей и сформулировать критерии успешности системы.

При разработке архитектуры мультиагентной системы применяется методология итеративного проектирования и прототипирования, опирающаяся на принципы микросервисной архитектуры~\cite{newman2021building}. Для проектирования взаимодействия между агентами используется подход, основанный на исследованиях в области оркестрации языковых моделей~\cite{shen2023hugginggpt}, с учетом специфики задачи автоматизации роли вердера.

Для оценки эффективности языковых моделей в контексте генерации игрового контента применяются методы, заимствованные из области оценки генеративных систем~\cite{zhou2023evaluation}, включая как количественные метрики (согласованность, соответствие тематике, информационная точность), так и качественные критерии (нарративная убедительность, игровая ценность).

В процессе разработки прототипа и его усовершенствованной версии используется методология Agile с короткими итерациями и постоянным взаимодействием с конечными пользователями~\cite{martin2019clean}. Это позволяет оперативно вносить изменения в архитектуру и функциональность системы на основе получаемой обратной связи.

Для преодоления ограничений языковых моделей, в частности проблемы галлюцинаций, применяется методология RAG (Retrieval-Augmented Generation)~\cite{lewis2020retrieval}, адаптированная к специфике игрового контекста с динамически меняющейся информационной базой.

Тестирование системы проводится с использованием методов качественной оценки пользовательского опыта~\cite{albert2013measuring}, включая структурированные опросы, интервью и наблюдение за игровыми сессиями. Для количественной оценки применяются метрики вычислительной эффективности, точности и согласованности генерируемого контента.

Таким образом, методология исследования представляет собой комплексный подход, сочетающий теоретические и эмпирические методы, что обеспечивает всестороннее изучение проблемы автоматизации ВПИ с помощью мультиагентных систем на основе языковых моделей.
