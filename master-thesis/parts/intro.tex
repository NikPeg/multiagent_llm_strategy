Современные достижения в области искусственного интеллекта и, в частности, больших языковых моделей (LLM) открывают новые перспективы для традиционных интерактивных развлечений~\cite{brown2020language, ouyang2022training}. Одной из областей, где применение искусственного интеллекта имеет значительный потенциал, являются текстовые стратегические игры, такие как военно-политические игры (ВПИ) — жанр, сочетающий элементы стратегии, ролевой игры и коллективного сторителлинга~\cite{catalogwpg2023}.

ВПИ представляют собой текстовые игры, в которых игроки управляют сложными структурами (государствами, военными силами, политическими организациями) путем написания приказов, а судья (или вердер) оценивает эти приказы и формирует вердикты — текстовые описания результатов действий~\cite{wpg-definition}. Этот процесс требует от судьи глубокого понимания игрового мира, механик, а также способности генерировать связные и логичные повествования, что делает эту роль одной из самых трудоемких в организации игры. Кроме того, традиционно судья ограничен в скорости обработки приказов, что создает естественный «потолок» для темпа игры и количества участников.

Большие языковые модели, такие как GPT-4, демонстрируют впечатляющие способности к пониманию контекста, следованию инструкциям и генерации связных текстов~\cite{openai2023gpt4}. Эти характеристики потенциально позволяют им выполнять роль судьи в ВПИ, автоматизируя процесс создания вердиктов и значительно ускоряя игровой процесс. Однако использование LLM в таком качестве сопряжено с рядом технических и методологических вызовов, включая проблему галлюцинаций, ограничения контекстного окна и сложности в поддержании долговременной согласованности~\cite{bubeck2023sparks, liu2023evaluating}.

В данной работе представлена разработка мультиагентной текстовой стратегической игры на основе оркестрируемых языковых моделей — системы, использующей несколько специализированных LLM-агентов для выполнения различных функций судьи в ВПИ. Исследование включает как теоретическое обоснование подхода, так и практическую реализацию в виде работающего прототипа, протестированного реальными игроками. Особое внимание уделяется механизмам обеспечения целостности игрового мира, преодоления ограничений LLM и создания интуитивно понятного пользовательского опыта.

Работа основывается на междисциплинарном подходе, объединяющем методы искусственного интеллекта, игрового дизайна и нарративных исследований~\cite{yuan2022wordcraft, park2023generative}. Представленная система не только демонстрирует практическое применение современных LLM в новой предметной области, но и открывает перспективы для создания более масштабных и динамичных текстовых игр, доступных широкой аудитории.

\subsection{Актуальность темы}

Текстовые стратегические игры жанра военно-политических игр (ВПИ) занимают особую нишу в игровой индустрии, предоставляя уникальный опыт коллективного стратегического взаимодействия. Ключевым ограничением данного жанра является высокая зависимость от человека-вердера (судьи), который обрабатывает игровые ситуации и формирует нарративную основу игры~\cite{dtf2021}. Это делает организацию подобных игр трудоемкой, снижает их доступность и ограничивает масштабы игрового сообщества~\cite{dtf2021}.

Современное развитие больших языковых моделей (LLM) создает предпосылки для решения данной проблемы~\cite{cloudru2025}. Недавние исследования показывают, что автоматизация создания интерактивного контента с помощью ИИ может существенно изменить подход к разработке игр с текстовой основой~\cite{uniteai2025}. Особую актуальность приобретает архитектурный подход на основе мультиагентных систем, где различные аспекты игрового взаимодействия обрабатываются специализированными ИИ-агентами~\cite{wikipediaMAS2025}. Подобная оркестрация позволяет преодолеть ограничения единичных моделей через разделение обязанностей и специализацию~\cite{wikipediaMAS2025}. В случае ВПИ это особенно важно, поскольку игра требует компетенций в различных областях: политике, экономике, военном деле, дипломатии~\cite{dtf2021}.

Разработка мультиагентной системы для ВПИ также имеет значение в контексте растущего интереса к цифровой гуманитаристике и инструментам совместного повествования~\cite{wikipediaDH2025}. Исследования показывают, что комбинация человеческого творчества и ИИ-ассистирования открывает новые горизонты для коллективного творчества и обмена идеями~\cite{wikipediaAIArt2025}.

С практической точки зрения, создание автоматизированной системы для проведения ВПИ может возродить интерес к жанру, сделать его доступным для более широкой аудитории и заложить основу для новых форм социального взаимодействия в цифровых средах~\cite{uniteai2025}. В научном плане проект представляет ценность как исследование применимости мультиагентных систем для поддержания сложных нарративных структур и последовательных игровых вселенных~\cite{habrMAS2018}.

Таким образом, разработка мультиагентной системы на основе оркестрируемых языковых моделей для автоматизации ВПИ представляет собой актуальную задачу, решение которой способно обогатить как теорию искусственного интеллекта, так и практику игрового дизайна.