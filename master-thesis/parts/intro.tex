Современные достижения в области искусственного интеллекта и, в частности, больших языковых моделей (LLM) открывают новые перспективы для традиционных интерактивных развлечений~\cite{brown2020language, ouyang2022training}. Одной из областей, где применение искусственного интеллекта имеет значительный потенциал, являются текстовые стратегические игры, такие как военно-политические игры (ВПИ) — жанр, сочетающий элементы стратегии, ролевой игры и коллективного сторителлинга~\cite{catalogwpg2023}.

ВПИ представляют собой текстовые игры, в которых игроки управляют сложными структурами (государствами, военными силами, политическими организациями) путем написания приказов, а судья (или вердер) оценивает эти приказы и формирует вердикты — текстовые описания результатов действий~\cite{wpg-definition}. Этот процесс требует от судьи глубокого понимания игрового мира, механик, а также способности генерировать связные и логичные повествования, что делает эту роль одной из самых трудоемких в организации игры. Кроме того, традиционно судья ограничен в скорости обработки приказов, что создает естественный «потолок» для темпа игры и количества участников.

Большие языковые модели, такие как GPT-4, демонстрируют впечатляющие способности к пониманию контекста, следованию инструкциям и генерации связных текстов~\cite{openai2023gpt4}. Эти характеристики потенциально позволяют им выполнять роль судьи в ВПИ, автоматизируя процесс создания вердиктов и значительно ускоряя игровой процесс. Однако использование LLM в таком качестве сопряжено с рядом технических и методологических вызовов, включая проблему галлюцинаций, ограничения контекстного окна и сложности в поддержании долговременной согласованности~\cite{bubeck2023sparks, liu2023evaluating}.

В данной работе представлена разработка мультиагентной текстовой стратегической игры на основе оркестрируемых языковых моделей — системы, использующей несколько специализированных LLM-агентов для выполнения различных функций судьи в ВПИ. Исследование включает как теоретическое обоснование подхода, так и практическую реализацию в виде работающего прототипа, протестированного реальными игроками. Особое внимание уделяется механизмам обеспечения целостности игрового мира, преодоления ограничений LLM и создания интуитивно понятного пользовательского опыта.

Работа основывается на междисциплинарном подходе, объединяющем методы искусственного интеллекта, игрового дизайна и нарративных исследований~\cite{yuan2022wordcraft, park2023generative}. Представленная система не только демонстрирует практическое применение современных LLM в новой предметной области, но и открывает перспективы для создания более масштабных и динамичных текстовых игр, доступных широкой аудитории.
