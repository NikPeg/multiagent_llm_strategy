На основе анализа теоретических аспектов военно-политических игр, возможностей и ограничений современных языковых моделей, а также существующих подходов к автоматизации геймплея, можно сформулировать комплексную задачу разработки мультиагентной текстовой стратегической игры на основе оркестрируемых языковых моделей. Данная глава детализирует функциональные и нефункциональные требования к разрабатываемой системе, описывает ключевых акторов и сценарии взаимодействия с системой, а также представляет архитектурные решения, обеспечивающие эффективное функционирование игры. Особое внимание уделяется специфике интеграции языковых моделей для автоматизации роли вердера с учетом выявленных ограничений и оптимальных методов их преодоления. Представленные в данной главе решения служат основой для последующей разработки как первичного прототипа системы, так и её улучшенной версии.
\subsection{Анализ требований пользователей}

Для разработки эффективной мультиагентной текстовой стратегической игры на основе оркестрируемых языковых моделей критически важно понимание ожиданий, предпочтений и потребностей целевой аудитории. В рамках исследования было проведено детальное интервьюирование потенциальных пользователей (подробные результаты представлены в Приложении~\ref{appendix:1}), что позволило выявить ключевые требования и сформировать целостное представление о желаемых характеристиках системы.

\subsubsection{Платформа и пользовательский интерфейс}

Анализ предпочтений пользователей относительно платформы проведения игры показал явное преобладание Telegram как предпочтительной платформы. Около 50\% респондентов однозначно высказались за использование данного мессенджера, аргументируя свой выбор удобством разметки, лучшей системой уведомлений и более интуитивным интерфейсом. Примерно 30\% опрошенных не выразили явных предпочтений между различными платформами, в то время как меньшинство (около 20\%) предпочли бы использование ВКонтакте.

Исходя из этих данных, Telegram определяется как основная платформа для разработки игрового интерфейса, с возможным последующим расширением на другие платформы.

\subsubsection{Монетизация и игровой темп}

Вопрос монетизации игры выявил значительную дифференциацию в готовности пользователей оплачивать игровые услуги:

\begin{itemize}
    \item Около 40\% опрошенных выразили готовность платить за игровые услуги в диапазоне от 300 до 2000 рублей в месяц, с преобладанием ценового диапазона 300-500 рублей.

    \item Оставшиеся 60\% предпочли бы бесплатный формат, с возможностью единоразовых пожертвований.
\end{itemize}

Анализ предпочтений относительно темпа игры выявил корреляцию между готовностью платить и желаемой частотой получения вердиктов. Пользователи, готовые платить большие суммы, ожидают более высокого темпа игры (до нескольких вердиктов в день), в то время как сторонники бесплатного формата удовлетворены более медленным темпом (около одного вердикта в день или реже).

На основе этих данных рекомендуется разработка многоуровневой системы монетизации:

\begin{enumerate}
    \item Бесплатный базовый уровень с ограниченным количеством вердиктов (1 в день или реже).

    \item Средний платный уровень (300-500 рублей в месяц) с умеренным темпом игры (2-3 верда в день).

    \item Премиум-уровень (1000-2000 рублей в месяц) с высоким темпом игры и дополнительными возможностями.
\end{enumerate}

\subsubsection{Игровые механики и особенности}

\paragraph{Отношение к элементам случайности}

Анализ отношения игроков к использованию случайности в игровом процессе выявил следующую картину:

\begin{itemize}
    \item Около 60\% респондентов выразили положительное отношение к включению элементов случайности в игровой процесс, предпочитая баланс между предопределенными результатами и случайными событиями. Многие указали на желательное соотношение 7:3, где 70\% результатов определяются действиями игрока, а 30\% — случайными факторами.

    \item Примерно 30\% выразили нейтральное отношение, не имея сильных предпочтений.

    \item Лишь 10\% участников высказались против использования случайности в игре, допуская её только в редких случаях (например, катаклизмы или чрезвычайные события).
\end{itemize}

Учитывая эти предпочтения, рекомендуется реализация настраиваемой системы случайности с возможностью регулирования её влияния на игровой процесс (параметр temperature в запросах к языковой модели) и обеспечения прозрачности в определении результатов действий.

\paragraph{Боевая система}

Предпочтения относительно боевой системы демонстрируют разнообразие подходов:

\begin{itemize}
    \item Большинство игроков предпочитают стратегический уровень принятия решений, где они определяют общие направления действий, а не управляют каждым юнитом напрямую.

    \item Некоторые игроки выразили желание иметь возможность выбора между детальным управлением и стратегическими приказами в зависимости от ситуации.

    \item Отдельные участники высказались за использование систем на основе кубиков (по аналогии с D\&D) для определения результатов сражений.
\end{itemize}

Оптимальным решением представляется создание гибкой боевой системы, позволяющей игрокам выбирать уровень детализации своего участия в военных действиях, с сохранением акцента на стратегических решениях и их нарративной интерпретации.

\paragraph{Визуализация и мультимедиа}

Относительно визуальных элементов игры выявлены следующие предпочтения:

\begin{itemize}
    \item Около 50\% респондентов проявили интерес к автоматической генерации изображений, иллюстрирующих значимые события и результаты действий.

    \item Большинство игроков (около 70\%) положительно отнеслись к использованию графиков и диаграмм для отображения экономических показателей, военной мощи и других количественных характеристик.

    \item Примерно 30\% опрошенных выразили интерес к функции расшифровки голосовых сообщений для упрощения коммуникации.
\end{itemize}

Рекомендуется внедрение опциональных визуальных элементов, дополняющих основной текстовый формат игры, с акцентом на информативности и эстетической ценности.

\subsubsection{Отношение к использованию языковых моделей}

Анализ отношения игроков к применению языковых моделей в роли вердера показал следующие результаты:

\begin{itemize}
    \item Около 40\% респондентов выразили положительное отношение к использованию ИИ для автоматизации вердинга, отмечая потенциальные преимущества в скорости и доступности.

    \item Примерно 30\% заняли нейтральную позицию, не видя принципиальных препятствий, но выражая определенные опасения относительно качества и последовательности генерируемого контента.

    \item Около 30\% высказали негативное отношение, опасаясь потери человеческого творческого элемента и уникального стиля, присущего человеку-вердеру.
\end{itemize}

Многие участники, независимо от общего отношения, выразили беспокойство относительно способности языковых моделей сохранять контекст игры, учитывать специфику игрового мира и обеспечивать последовательность нарратива. Часто высказывалось пожелание сохранить роль человека-администратора для надзора и корректировки действий ИИ.

\subsubsection{Коммуникационные предпочтения}

В вопросе организации коммуникации между игроками мнения разделились:

\begin{itemize}
    \item Около 60\% предпочитают общение через игрового бота, что обеспечивает сохранение всех коммуникаций в общем контексте игры и потенциально увеличивает иммерсивность.

    \item Примерно 40\% высказались за использование традиционных каналов коммуникации (групповые чаты, личные сообщения), обеспечивающих более непосредственное взаимодействие.
\end{itemize}

Некоторые игроки также выразили заинтересованность в механизмах, обеспечивающих анонимность коммуникации для предотвращения "метаигры" — ситуаций, когда личные отношения между игроками влияют на игровой процесс.

\subsubsection{Ключевые выводы и приоритетные требования}

На основе проведенного анализа можно сформулировать следующие ключевые требования к разрабатываемой системе:

\begin{enumerate}
    \item Реализация интерфейса на платформе Telegram с интуитивно понятной навигацией и системой уведомлений.

    \item Создание многоуровневой системы монетизации, обеспечивающей различные темпы игры в зависимости от выбранного тарифа.

    \item Внедрение настраиваемых элементов случайности с преобладанием детерминированных результатов, зависящих от действий игрока.

    \item Разработка гибкой боевой системы с акцентом на стратегическом уровне принятия решений.

    \item Интеграция визуальных элементов (графики, диаграммы, генерируемые изображения) как дополнение к основному текстовому контенту.

    \item Обеспечение механизмов сохранения контекста и последовательности игрового мира при использовании языковых моделей.

    \item Сохранение роли человека-администратора для надзора и корректировки автоматически генерируемого контента.

    \item Создание интегрированной системы коммуникации с возможностью выбора между игровыми и внеигровыми каналами общения.

    \item Разработка механизмов для баланса между автоматизацией и сохранением уникального игрового опыта.
\end{enumerate}

Эти требования являются фундаментом для дальнейшей разработки функциональных и нефункциональных спецификаций системы и определяют основные направления проектирования архитектуры мультиагентной системы для автоматизации ВПИ.

\subsection{Функциональные требования к системе}

\subsection{Нефункциональные требования}

\subsection{Диаграмма прецедентов и сценарии использования}

\subsection{Требования к пользовательскому интерфейсу}

\subsection{Стратегия обработки ошибок и исключительных ситуаций}

\subsection{Масштабируемость и производительность}

\subsection{Этические аспекты и ограничения системы}