На основе анализа теоретических аспектов военно-политических игр, возможностей и ограничений современных языковых моделей, а также существующих подходов к автоматизации геймплея, можно сформулировать комплексную задачу разработки мультиагентной текстовой стратегической игры на основе оркестрируемых языковых моделей. Данная глава детализирует функциональные и нефункциональные требования к разрабатываемой системе, описывает ключевых акторов и сценарии взаимодействия с системой, а также представляет архитектурные решения, обеспечивающие эффективное функционирование игры. Особое внимание уделяется специфике интеграции языковых моделей для автоматизации роли вердера с учетом выявленных ограничений и оптимальных методов их преодоления. Представленные в данной главе решения служат основой для последующей разработки как первичного прототипа системы, так и её улучшенной версии.
\subsection{Анализ требований пользователей}

\subsection{Функциональные требования к системе}

\subsection{Нефункциональные требования}

\subsection{Диаграмма прецедентов и сценарии использования}

\subsection{Требования к пользовательскому интерфейсу}

\subsection{Стратегия обработки ошибок и исключительных ситуаций}

\subsection{Масштабируемость и производительность}

\subsection{Этические аспекты и ограничения системы}