В рамках данной курсовой работы была разработана и реализована система диалогового агента для проведения многопользовательских ролевых игр в сеттинге древнего мира. Исследование прошло несколько последовательных этапов, начиная от анализа предметной области и существующих решений, формулировки требований, разработки прототипа и заканчивая созданием финальной версии системы.

Основным результатом работы стало создание двух версий системы, реализующих различные подходы к автоматизации роли вердера в военно-политических играх:

\begin{enumerate}
\item \textbf{Прототип системы} — первичная реализация с фокусом на проверку концепции мультиагентного подхода, позволившая выявить ключевые проблемы и ограничения, такие как сложности с поддержанием долгосрочной контекстной памяти и балансировкой игрового процесса.

\item \textbf{Финальная версия системы} — улучшенная реализация на базе оптимально подобранной языковой модели DeepSeek-R1-Distill-Qwen-32B с акцентом на асинхронную обработку запросов и контекстно-зависимую генерацию ответов через механизм RAG.
\end{enumerate}

В процессе разработки были решены следующие ключевые задачи:

\begin{itemize}
\item Проведен комплексный анализ и сравнение различных языковых моделей для определения оптимального баланса между качеством генерации и вычислительными требованиями

\item Разработана архитектура системы, обеспечивающая асинхронную обработку запросов множества пользователей и сохранение состояния игрового мира

\item Реализован механизм контекстно-зависимой генерации ответов с учетом особенностей конкретного государства и исторического сеттинга

\item Создана система фильтрации анахронизмов для поддержания исторической достоверности игрового мира

\item Разработан пользовательский интерфейс на базе платформы Telegram, обеспечивающий интуитивно понятное взаимодействие с системой

\item Проведено тестирование с реальными пользователями и внесены улучшения на основе полученной обратной связи
\end{itemize}

Результаты проведенного исследования демонстрируют перспективность использования современных языковых моделей для автоматизации роли вердера в военно-политических играх. Созданная система позволяет существенно снизить трудозатраты на проведение игровых сессий, сохраняя при этом высокий уровень иммерсивности и реалистичности игрового процесса.

Дальнейшее развитие проекта может быть направлено на несколько ключевых направлений: улучшение межпользовательского взаимодействия, расширение игрового мира, совершенствование боевой системы для повышения зрелищности и стратегической глубины игрового процесса. Особенно перспективным представляется интеграция разработанной системы в существующие крупные проекты военно-политических игр с целью получения профессиональной обратной связи от опытных вердеров и игроков, а также обмена опытом с профессионалами жанра для дальнейшего совершенствования технологии и игровых механик. Такое сотрудничество позволит не только улучшить текущую реализацию, но и способствовать формированию новых стандартов в области автоматизации ролевых игр с использованием генеративных языковых моделей.