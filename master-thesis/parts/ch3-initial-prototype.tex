В данной главе рассматривается процесс разработки и тестирования первичного прототипа мультиагентной текстовой стратегической игры RELOAD WPG, реализованного на основе языковой модели GPT-4. Прототип представлял собой экспериментальную платформу, позволившую проверить фундаментальную гипотезу о возможности автоматизации роли вердера с использованием современных языковых моделей. Система, интегрированная с мессенджером Telegram, включала набор специализированных агентов для обработки приказов игроков, управления проектами и межгосударственного взаимодействия. Тестовая игровая сессия с реальными пользователями позволила выявить ключевые преимущества и ограничения выбранного подхода, а также собрать ценную обратную связь для разработки улучшенной версии системы. Несмотря на ряд технических несовершенств, прототип продемонстрировал потенциальную жизнеспособность концепции и подтвердил возможность использования языковых моделей для автоматизации вердерства в ВПИ.
\subsection{Цели и архитектура прототипа}

Разработка первичного прототипа RELOAD WPG преследовала несколько ключевых целей, определивших его функциональность и архитектурные решения. Основополагающей целью выступала практическая проверка возможности использования языковых моделей для автоматизации роли вердера в военно-политических играх — гипотеза, которая до этого момента не имела полноценной практической проверки в сообществе ВПИ.

\subsubsection{Основные цели прототипа}

\begin{itemize}
    \item \textbf{Проверка фундаментальной гипотезы} о возможности использования современных языковых моделей для генерации качественных вердиктов, соответствующих ожиданиям игроков в жанре ВПИ.

    \item \textbf{Создание доступной игровой платформы} для проведения полноценной игровой сессии с реальными пользователями, знакомыми с жанром ВПИ и имеющими опыт участия в "классических" играх с человеком-вердером.

    \item \textbf{Воссоздание атмосферы оригинальной "Внеземной ВПИ"} — успешного проекта, проведённого автором в 2017 году, с акцентом на наиболее драматичном историческом периоде этой игры (80-летняя война).

    \item \textbf{Сбор детальной обратной связи от пользователей} для выявления ограничений выбранного подхода и формирования требований к улучшенной версии системы.

    \item \textbf{Тестирование различных подходов к оркестрации языковых моделей} для выполнения специализированных функций в контексте игрового процесса.
\end{itemize}

\subsubsection{Ключевые задачи}

Для достижения поставленных целей были сформулированы следующие технические и организационные задачи:

\begin{enumerate}
    \item Разработать минимально жизнеспособную архитектуру системы, включающую базовые компоненты для обработки приказов, генерации вердиктов и управления игровым состоянием.

    \item Интегрировать систему с платформой Telegram для обеспечения удобного взаимодействия игроков с игрой через привычный мессенджер.

    \item Настроить языковую модель GPT-4 для генерации содержательных и стилистически соответствующих вердиктов в контексте научно-фантастического сеттинга.

    \item Реализовать механизм проектов, позволяющий игрокам инициировать долгосрочные действия с отложенными результатами.

    \item Создать систему межгосударственного взаимодействия, обеспечивающую влияние действий одних игроков на другие государства.

    \item Обеспечить базовую эмуляцию экономического и технологического развития государств на основе принимаемых игроками решений.

    \item Разработать боевую систему для моделирования военных конфликтов между игроками.

    \item Организовать коммуникационные каналы между игроками (чат, новостной канал) для обеспечения социального взаимодействия.
\end{enumerate}

\subsubsection{Архитектурные принципы прототипа}

При проектировании архитектуры первичного прототипа был принят ряд архитектурных решений, отражающих экспериментальный характер системы и необходимость быстрой итерации:

\begin{itemize}
    \item \textbf{Модульность} — система строилась как набор относительно независимых компонентов (агентов), каждый из которых отвечал за определённый аспект игрового процесса и мог быть модифицирован или заменён без серьёзного влияния на другие части системы.

    \item \textbf{Минимализм} — в первичный прототип включались только те функции, которые были необходимы для проверки основной гипотезы, с минимумом дополнительных возможностей.

    \item \textbf{Гибкость конфигурации} — система проектировалась с возможностью быстрой корректировки параметров в процессе тестирования, включая настройки языковой модели, вероятностные параметры игровых механик и шаблоны промптов.

    \item \textbf{Централизованное управление игровым состоянием} — для обеспечения согласованности игрового мира использовалась централизованная структура хранения данных о текущем состоянии игры и истории взаимодействий.

    \item \textbf{Приоритет взаимодействия через естественный язык} — интерфейс системы был спроектирован с акцентом на свободную текстовую коммуникацию, минимизируя необходимость использования специальных команд или форматированных запросов.
\end{itemize}
\begin{figure}[h]
    \centering
    \includegraphics[width=\textwidth]{figures/agents.png}
    \caption{Основные компоненты системы и их взаимодействие}
    \label{fig:agents}
\end{figure}\\
\subsubsection{Ограничения прототипа}

Для первичного прототипа были приняты следующие сознательные ограничения:

\begin{itemize}
    \item \textbf{Ограниченное количество игроков} — прототип был рассчитан на участие до 15 игроков, что позволяло обеспечить комфортную нагрузку на систему и внимательно отслеживать процесс тестирования.

    \item \textbf{Фиксированный сеттинг} — в отличие от некоторых ВПИ с процедурно генерируемым миром, RELOAD WPG использовал заранее определённый научно-фантастический сеттинг, основанный на вселенной оригинальной "Внеземной ВПИ".

    \item \textbf{Упрощённая экономическая модель} — для первого прототипа была использована базовая экономическая система, не включающая детальное моделирование торговых отношений, ресурсных цепочек и сложных финансовых механизмов.

    \item \textbf{Зависимость от внешних API} — система использовала облачный API OpenAI для доступа к GPT-4, что накладывало ограничения на скорость обработки запросов и стоимость эксплуатации.
\end{itemize}

Таким образом, первичный прототип RELOAD WPG был спроектирован как целенаправленный эксперимент, призванный проверить жизнеспособность концепции автоматизированного вердерства с использованием языковых моделей и сформировать основу для более совершенной системы на основе полученного опыта и обратной связи пользователей.

\subsection{Интеграция с Telegram и выбор языковой модели}

Выбор платформы для реализации пользовательского интерфейса и интеграция с языковой моделью являлись критически важными аспектами разработки первичного прототипа, определяющими доступность, удобство использования и технические возможности системы. Для прототипа была выбрана комбинация мессенджера Telegram в качестве интерфейсной платформы и языковой модели GPT-4 от OpenAI в качестве основного генеративного компонента.

\subsubsection{Интеграция с Telegram}

Telegram был выбран в качестве платформы для разработки пользовательского интерфейса по ряду причин:

\begin{itemize}
    \item \textbf{Широкая распространённость} среди целевой аудитории ВПИ, подтверждённая результатами предварительного опроса потенциальных игроков.

    \item \textbf{Развитый Bot API}, предоставляющий необходимые инструменты для создания интерактивных ботов с поддержкой кнопок, форматированного текста и медиафайлов.

    \item \textbf{Экосистема каналов и групп}, позволяющая организовать разные аспекты игрового взаимодействия (личные сообщения, новостная лента, общий чат) в рамках единой платформы.

    \item \textbf{Кроссплатформенность}, обеспечивающая доступ к игре с различных устройств (мобильные телефоны, планшеты, компьютеры) без необходимости разработки отдельных клиентских приложений.
\end{itemize}

Архитектура интеграции с Telegram включала следующие компоненты:

\begin{enumerate}
    \item \textbf{Основной игровой бот} (@reload\_wpg\_bot) — центральный элемент взаимодействия, через который игроки отправляли приказы, получали вердикты и управляли своими государствами.

    \item \textbf{Новостной канал} (@reload\_wpg\_news) — публичный канал для распространения информации о ключевых событиях в игровом мире, правилах и обновлениях системы.

    \item \textbf{Общий чат} (@reload\_wpg\_chat) — групповой чат для общения между участниками, обсуждения игровых событий и неформального взаимодействия.
\end{enumerate}

Пользовательский интерфейс бота был спроектирован с учётом специфики текстовых стратегических игр и включал следующие ключевые элементы:

\begin{itemize}
    \item \textbf{Процесс регистрации и онбординга} — при первом запуске бота пользователю предлагалось пройти короткую процедуру инициализации, включающую выбор страны из 9 доступных (соответствующих государствам из оригинальной "Внеземной ВПИ") и выбор уровня подписки (от бесплатного до премиум).

    \item \textbf{Клавиатура с основными функциями} — постоянно доступный набор кнопок, включающий доступ к карте, графикам, проектам, информации о стране, настройкам погружения и функции отправки телеграмм другим игрокам.

    \item \textbf{Текстовый ввод} — основной метод взаимодействия, позволяющий игрокам формулировать приказы, задавать вопросы и инициировать проекты в свободной форме, без необходимости следования строгим синтаксическим правилам.
\end{itemize}

Одной из технических особенностей реализации стало создание механизма, позволяющего администратору игры отправлять сообщения через бота от его имени, что обеспечивало возможность ручного вмешательства в процесс игры при необходимости корректировки генерируемого контента.

\subsubsection{Выбор и настройка языковой модели}

В качестве основы для генерации вердиктов и ответов на вопросы игроков была выбрана языковая модель GPT-4 от OpenAI, доступная через API. Этот выбор был обусловлен следующими факторами:

\begin{itemize}
    \item \textbf{Превосходное качество генерации} длинных связных текстов с сохранением контекста, что критически важно для создания вердиктов в жанре ВПИ.

    \item \textbf{Способность к условному рассуждению} и логической связности при генерации событий, обеспечивающая реалистичность и непротиворечивость игрового нарратива.

    \item \textbf{Гибкость в настройке} через систему промптов, позволяющая адаптировать модель к специфическим требованиям игрового контекста без необходимости дополнительного обучения.

    \item \textbf{Доступность через API} с предсказуемой моделью тарификации, что упрощало разработку и позволяло прогнозировать эксплуатационные расходы.
\end{itemize}

Для обеспечения оптимального функционирования модели в контексте ВПИ были применены следующие подходы к её настройке:

\begin{enumerate}
    \item \textbf{Разработка базового системного промпта}, определяющего общий контекст игры, сеттинг, технологические ограничения эпохи и ключевые механики взаимодействия.

    \item \textbf{Создание специализированных промптов} для различных типов агентов (обработчик событий, генератор отчётов о странах, ассистент проектов, статистический анализатор), оптимизирующих модель для выполнения конкретных функций.

    \item \textbf{Настройка параметра температуры} для обеспечения баланса между детерминированностью и творческим разнообразием. Для большинства операций использовалось значение temperature = 0.7, обеспечивающее достаточную вариативность при сохранении логической согласованности.

    \item \textbf{Экспериментальная валидация промптов} с использованием различных формулировок и структур для выявления наиболее эффективных подходов к инструктированию модели.
\end{enumerate}

В процессе тестирования прототипа, в ответ на рост эксплуатационных расходов, связанных с интенсивным использованием модели, особенно во время военных действий когда стоимость достигла \$10 в сутки, была осуществлена миграция на более экономичную модель GPT-4o mini. Примечательно, что, согласно отзывам игроков, это изменение не привело к заметному снижению качества генерируемых вердиктов, что свидетельствует о потенциальной возможности использования более компактных и эффективных моделей в будущих итерациях системы.

\subsubsection{Интерфейс взаимодействия и правила игры}

Для облегчения вхождения пользователей в игровой процесс были разработаны подробные правила и руководство по использованию бота, доступные через новостной канал. Основные положения включали:

\begin{itemize}
    \item \textbf{Общее описание} игры, её жанра, сеттинга и временной линии, начинающейся в 2650 году, за несколько лет до знакового события оригинальной ВПИ — Восьмидесятилетней войны.

    \item \textbf{Уровни погружения} — система подписок, определяющая доступные функции и частоту взаимодействия:
    \begin{itemize}
        \item Уровень 1 (бесплатный) — базовый доступ с ограниченной частотой приказов
        \item Уровень 2 (150 рублей в неделю) — дополнительные функции и повышенная частота взаимодействия
        \item Уровень 3 (450 рублей в неделю) — максимальные возможности, включая доступ к графикам и наиболее высокая частота взаимодействия
    \end{itemize}

    \item \textbf{Специальные правила} — важные оговорки, такие как возможность администратора вмешиваться в игру, приоритет вердиктов администратора над автоматически генерируемыми и даже толерантность к "prompt injection" в развлекательных целях.

    \item \textbf{Инструкция по использованию бота} — детальное описание доступных функций и способов взаимодействия, включая отправку приказов, просмотр карты, управление проектами, отправку телеграмм другим правителям и доступ к аналитическим графикам.
\end{itemize}

Система кнопок в интерфейсе бота предоставляла доступ к ключевым функциям:

\begin{itemize}
    \item \textbf{Карта} — визуальное представление мира с актуальными границами и контролируемыми территориями
    \item \textbf{Графики} — аналитические данные о динамике ВВП, лояльности и численности населения (доступны с III уровня погружения)
    \item \textbf{Проекты} — список активных долгосрочных инициатив и их статус готовности
    \item \textbf{Страна} — информация о текущем государстве игрока и возможность смены страны (функция находилась в разработке)
    \item \textbf{Телеграмма} — функция отправки сообщений правителям других стран (доступна со II уровня погружения)
    \item \textbf{Погружение} — управление уровнем подписки и соответствующими возможностями
\end{itemize}

Реализованный интерфейс обеспечивал баланс между удобством использования и функциональностью, однако уже на этапе первичного тестирования выявились определённые недостатки, такие как недостаточная заметность ссылки на новостной канал в приветственном сообщении, что потребовало дополнительных коммуникаций с игроками. Эти наблюдения стали важной частью обратной связи для планирования улучшений интерфейса в последующих версиях системы.

\subsection{Система агентов и обработка приказов}

Ядром функциональности первичного прототипа RELOAD WPG являлась система специализированных агентов, отвечающих за обработку пользовательских запросов и поддержание согласованного состояния игрового мира. Архитектура системы была построена по принципу разделения ответственности, где каждый агент выполнял определённую функцию в цепочке обработки информации.

\subsubsection{Архитектура системы агентов}

Система включала четыре основных агента, взаимодействующих между собой и с хранилищем данных:

\begin{enumerate}
    \item \textbf{User Event Handler (Обработчик пользовательских событий)} — центральный агент, принимающий входящие сообщения от пользователей, классифицирующий их и определяющий дальнейший маршрут обработки. Этот агент выполнял следующие ключевые функции:
    \begin{itemize}
        \item Определение типа сообщения (приказ, вопрос, запрос информации)
        \item Анализ приказа для выявления стран, на которые он может оказать влияние
        \item Идентификация долгосрочных проектов и оценка времени их выполнения
        \item Генерация случайных событий в конце игрового года
        \item Формирование структурированного JSON-представления обработанного приказа для дальнейшего использования другими агентами
        \item Генерация ежедневных новостей, которые публиковались в новостном канале в 19:30
    \end{itemize}

    \item \textbf{Country Report (Отчёт о стране)} — агент, отвечающий за генерацию уведомлений о влиянии действий одного государства на другие. Активировался в следующих случаях:
    \begin{itemize}
        \item Когда страна явно упоминалась в приказе или вердикте другого игрока
        \item Когда действие одного государства косвенно затрагивало интересы другого государства (с определённой вероятностью)
        \item При генерации реакций на значимые международные события
    \end{itemize}

    \item \textbf{Project Assistant (Ассистент проектов)} — агент, управляющий жизненным циклом долгосрочных проектов. Его функции включали:
    \begin{itemize}
        \item Отслеживание статуса выполнения проектов
        \item Генерацию итоговых отчётов по завершённым проектам в конце игрового года
        \item Обновление информации о проектах в базе данных
    \end{itemize}

    \item \textbf{Statistic Graph (Статистический график)} — агент, отвечающий за экономические и демографические расчёты. Функционал включал:
    \begin{itemize}
        \item Анализ экономического влияния действий игрока на основе информации из треда
        \item Расчёт изменений ВВП, лояльности и численности населения в конце игрового года
        \item Формирование данных для визуализации динамики ключевых показателей в виде графиков
        \item Хранение и обновление экономических показателей в базе данных
    \end{itemize}
\end{enumerate}

\subsubsection{Хранение и управление данными}

Для хранения игрового состояния была реализована простая, но эффективная система на основе JSON-файлов:

\begin{itemize}
    \item \textbf{Базовые данные о странах} — каждое государство имело связанную запись, содержащую основную информацию:
    \begin{itemize}
        \item Название и идентификатор
        \item Текущие значения ВВП, лояльности и численности населения
        \item Историю изменения этих показателей для построения графиков
    \end{itemize}

    \item \textbf{Журнал проектов} — структурированный список активных и завершённых проектов с атрибутами:
    \begin{itemize}
        \item Название и описание проекта
        \item Игрок-инициатор
        \item Год начала и ожидаемый год завершения
        \item Статус проекта
    \end{itemize}
\end{itemize}

История взаимодействий с пользователями сохранялась непосредственно в Telegram-чатах и дублировалась в административную беседу, куда пересылались все сообщения игроков и ответы системы. Это обеспечивало возможность мониторинга игрового процесса администраторами и служило дополнительным резервным хранилищем истории взаимодействий.

\subsubsection{Процесс обработки приказов}

Типичный жизненный цикл обработки приказа игрока включал следующую последовательность шагов:

\begin{enumerate}
    \item \textbf{Приём сообщения} — пользователь отправлял текстовый приказ через интерфейс Telegram-бота.

    \item \textbf{Первичный анализ} — User Event Handler анализировал сообщение с помощью GPT-4 для определения его типа, выявления затрагиваемых стран и идентификации возможных проектов.

    \item \textbf{Генерация вердикта} — на основе приказа и контекстной информации о текущем состоянии страны формировался запрос к GPT-4 для генерации вердикта, описывающего результаты выполнения приказа.

    \item \textbf{Отправка ответа пользователю} — сгенерированный вердикт отправлялся игроку через бота.

    \item \textbf{Обработка межгосударственного влияния} — если приказ затрагивал интересы других стран, Country Report генерировал соответствующие уведомления и отправлял их заинтересованным игрокам.

    \item \textbf{Регистрация проектов} — если приказ инициировал долгосрочный проект, информация о нём фиксировалась в базе данных с указанием ожидаемого срока завершения.

    \item \textbf{Обновление игрового состояния} — по итогам обработки приказа могли обновляться различные аспекты игрового состояния, включая экономические показатели.
\end{enumerate}

В конце каждого игрового года (который соответствовал нескольким дням реального времени) происходили дополнительные процессы:

\begin{itemize}
    \item Project Assistant проверял список проектов, выявлял завершённые в текущем году и генерировал итоговые отчёты по ним.

    \item Statistic Graph анализировал экономическое влияние всех действий за год и рассчитывал обновлённые значения ВВП, лояльности и численности населения.

    \item User Event Handler генерировал случайные события для каждой страны, добавляя элемент непредсказуемости и динамики в игровой процесс.
\end{itemize}

Кроме того, User Event Handler ежедневно в 19:30 генерировал и публиковал в новостном канале сводку основных событий, произошедших в игровом мире. Эти новости служили важным элементом формирования общей картины происходящего для всех игроков и способствовали созданию единого информационного пространства.

\subsubsection{Проблемы и ограничения первичной архитектуры}

В ходе тестирования прототипа были выявлены несколько ключевых ограничений выбранной архитектуры:

\begin{itemize}
    \item \textbf{Ограниченная контекстуальная память} — поскольку каждый запрос к GPT-4 обрабатывался независимо, система сталкивалась с трудностями в поддержании долгосрочной согласованности и запоминании специфических деталей игрового мира, не включённых явно в контекст запроса.

    \item \textbf{Проблемы с межгосударственным взаимодействием} — изначальная реализация, где каждый игрок имел отдельный "тред" для взаимодействия, приводила к изоляции действий игроков друг от друга. Это ограничение было частично преодолено путём объединения всех игроков в общий тред, однако генерация уведомлений о влиянии действий других стран часто воспринималась игроками как навязчивая и нереалистичная.

    \item \textbf{Нереалистичные временные оценки проектов} — языковая модель имела тенденцию назначать необоснованно длительные сроки даже для относительно простых проектов (например, "отправка послов" могла оцениваться в 10-15 лет), что негативно влияло на динамику игры и вызывало фрустрацию у игроков.

    \item \textbf{Галлюцинации и несогласованность} — модель иногда генерировала фактически неверную информацию или противоречила ранее установленным фактам, особенно при взаимодействии нескольких игроков в одном контексте.

    \item \textbf{Высокая стоимость эксплуатации} — интенсивное использование API GPT-4, особенно в периоды активных военных действий, приводило к значительным эксплуатационным расходам (до \$10 в сутки), что ставило под вопрос экономическую устойчивость выбранного подхода в долгосрочной перспективе.
\end{itemize}

Для адаптации к выявленным ограничениям в ходе игровой сессии были внедрены несколько тактических улучшений:

\begin{itemize}
    \item Введение "боевого режима", позволяющего обрабатывать военные приказы с минимальными задержками, минуя стандартный механизм проектов.

    \item Настройка вероятностных параметров для уведомлений о межгосударственном влиянии (50\% при явном упоминании страны и 5\% в иных случаях), что снизило частоту нерелевантных уведомлений.

    \item Оптимизация использования контекстного окна для включения наиболее релевантной исторической информации при генерации вердиктов.

    \item Переход на более экономичную модель GPT-4o mini для снижения эксплуатационных расходов, что, неожиданно, не привело к заметному снижению качества вердиктов по отзывам игроков.
\end{itemize}

Опыт разработки и тестирования первичной системы агентов и механизмов обработки приказов предоставил ценную информацию для проектирования улучшенной архитектуры в последующих версиях, с акцентом на более эффективное управление контекстом, реалистичное моделирование темпоральных аспектов игрового мира и экономичное использование языковых моделей.

\subsection{Механизмы проектов и межгосударственного взаимодействия}

\subsection{Проведение тестовой игровой сессии}

\subsection{Анализ обратной связи от игроков}

\subsection{Выявленные проблемы и ключевые выводы}