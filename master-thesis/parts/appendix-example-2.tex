\subsection{Что больше всего понравилось в игре?}

Концепция + в целом система крутая, хоть и требует доработок.

* неплохое consistency - ± получалось реализовать желаемые идеи, за счёт чего каждая страна действительно выглядела уникально
* Реализация самых безумных идей - это весело
* Отыгрыш (я получил огромное удовольствие от взаимодействия с другими игроками. Это зависит от игроков, но моё желание входить в роль поддерживали, и получилось вайбово. У нас литералли был с Шамшем проект договора в пдфке с пунктами, так ещё и с посредником в переговорах) (тут хочу отметить, что мне понравился именно отыгрыш, а не душниловка на уровне диванной политики)
* Двигающие сюжет события, подкидываемые игрой

Бот, новости, ивенты

Сам формат впи и скорость игры

Графики! Их наличие действительно добавляет то дивное измерение игры, на которое можно яростно фокусироваться.

Взаимодействие игроков и карта

Война

Технические новшества в виде бота и любопытные взаимодействия. Системы рейтинга и ВВП.

\subsection{Что больше всего НЕ понравилось в игре?}
Pay2win + нет одного источника правды + бот часто не присылал что-то что происходит с моей страной

* периодически возникало ощущение neural mess. Возможно, на более умной версии гптшки с этим было получше, но я чувствовал, как нахожусь в узком контексте модельки, которая реагирует больше на то, как я оформлю приказ, чем на то, что было раньше
* Из-за этого же - часть идей так и не удалось нормально реализовать, и они улетели в труху. Союз Пара, премия Золотой Гой, дирижабль с жабками, в принципе развитие аэротранспорта - бот реализовывал все это в рамках приказа, но дальше об этом никак не вспоминалось, и как будто бы (точнее, я в этом уверен на 95%) взаимодействие с другими странами тут было лишь у меня в голове

Рандом, галлюцинирующий гпт

Бот очень сильно тупит, Хуан-Фернандес воюет с ХФ и Мордор требует капитуляцию ТСГЕМ это конечно смешно, но погружение портит

Все проблемы так или иначе будут про отсутствие воспринимаемой связности. ВПИ -- это одна из тех штук, где взмах крыла бабочки вызывает ураганы. При игре с нейроботом это невозможно, пока не начинается что-то совсем шизофреническое. То есть, я не могу слегка изменять свою культуру.

Очень странная разбалансировка сложностей приказов. Задачи в духе "Последить за Мордором", "Написать мафиози из ФТК" занимали больше времени, чем "Разработать систему нейроинтерфейсов для слияния всех мыслей всех людей".

Ну и мой запрос с более калькуляторной версией не реализовали, но тут пынемаю, все остальные были против.

Что из-за gpt может произойти вообще что угодно. Прогресс несся быстрее, чем успевали следить. Вот только что были дирижабли, проходит неделя и на меня нацеливают орбитальные пушки

Шиза нейронки

Отсутствие ощущения цельности происходящего и платная основа.

\subsection{Что бы ты хотел увидеть в будущих ВПИ?}
Приложение отдельное) (могу помочь)
+ больший баланс относительно эпохи
+ может быть другая вселенная, чтобы не было одного проторенного курса действий, а было больше пространства для воображения

* маловато в целом механик для взаимодействия с другими странами. С кем я торгую, на кого я могу так влиять, от кого я завишу, с кем у нас есть культурные связи итд - я понимаю, что
* Больше вестей из других стран - иногда прямо информационный вакуум был

Новые версии чат гпт

Больше концпептуальной свободы, лично я как игрок был вынужден играть за фракцию, к которой в ориге был слабо причастен

Вероятно, пока что нужно прикладывать чуть больше мясного мешка. Бот неплохо зарекомендовал себя как справочно-вопросная система, но, кажется, должен существовать ещё некий отсек "Дизайна того, как выглядит игровой мир в целом вот прямо сейчас и что в нём важно", которым должен заниматься человек.

Чтобы не возникало ситуаций, когда с одной стороны гедониум заливает мир, а с другой -- санаторий достраивают.

В идеале личный верд

Чтобы нейронка выдавала одинаковые ответы на одинаковые вопросы от разных игроков об одном и том же событии. Так же нейронка должна как-то помнить хронологию событий

Что-то более похожее на ВПИ, а не на калькулятор с прикрученным флейвор-текстом, который сам не понимает что описывает.