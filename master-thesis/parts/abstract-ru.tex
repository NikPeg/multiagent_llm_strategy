Работа посвящена разработке мультиагентной текстовой стратегической игры на основе оркестрируемых языковых моделей и исследованию применимости современных LLM в качестве автоматизированного мастера игры в жанре военно-политических игр (ВПИ).

В работе рассмотрен полный цикл создания подобной системы, включая проектирование архитектуры мультиагентной среды, оркестрацию языковых моделей для выполнения различных игровых функций, а также анализ результатов пилотного запуска с реальными пользователями. Особое внимание уделяется методам преодоления типичных проблем языковых моделей в контексте игрового процесса: галлюцинации, сохранение долгосрочного контекста и вычислительные ограничения.

Результаты исследования демонстрируют потенциал современных LLM для автоматизации роли гейм-мастера в текстовых стратегических играх, выявляют ключевые технические и игровые ограничения существующих подходов, а также предлагают набор технических решений для повышения качества игрового опыта, включая применение систем на основе RAG и локальных моделей для снижения стоимости эксплуатации игровой системы.