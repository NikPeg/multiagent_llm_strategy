\documentclass{includes/Thesis}

\addbibresource{parts/refs.bib}

\begin{thesis}[
    title = Разработка мультиагентной текстовой стратегической игры на основе оркестрируемых языковых моделей,
    year = 2025,
    %
    authorGroup = ИПИИ,
    authorName = Н. С. Пеганов,
    %
    academicTeacherTitle = {доцент ФКН департамента программной инженерии, к.т.н.},
    academicTeacherName = С. Л. Макаров,
    %
    hasAcademicCoteacher = false,
    academicCoteacherTitle = преподаватель базовой кафедры\\<<Системное программирование>> ИСП РАН в НИУ ВШЭ,
    academicCoteacherName = А. Е. Волков,
    %
    isAcademic = false,
    UDC = 004.05,
    %
    hasConsultant = false,
    consultantTitle = младший научный сотрудник Института системного программирования РАН,
    consultantName = С. А. Поляков,
    %
    keywordsRu = мультиагентная система; языковые модели; оркестрация LLM; текстовые стратегические игры; искусственный интеллект в играх; RAG; автоматизация игрового мастера,
    keywordsEn = multi-agent system; large language models; LLM orchestration; text-based strategy games; artificial intelligence in games; RAG; automated game mastering,
]

    \setAbstractResource{parts/abstract-ru}{parts/abstract-en}

    \setTerminologyResource{parts/terminology}

    \setIntroResource{parts/intro}

    \addChapter{Предметная область проекта и анализ существующих решений}{parts/ch1-literature-review}
    \addChapter{Постановка задачи на разработку системы}{parts/ch2-task}
    \addChapter{Первичный прототип системы}{parts/ch3-initial-prototype}
    \addChapter{Финальная версия системы}{parts/ch4-enhanced-system}
    \addChapter{Обсуждение результатов}{parts/ch5-results-discussion}

    \setConclusionResource{parts/ch6-conclusion}

    \addAppendix{Пример приложения}{parts/appendix-example-1}
    \addAppendix{Ещё один пример приложения}{parts/appendix-example-2}

\end{thesis}
